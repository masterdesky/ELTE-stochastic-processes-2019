\section{} \label{sec:2}
\quest{Vizsgáljuk az előadáson tárgyalt 2 Ising spinből álló rendszer relaxációjának problémáját! Az előadáson megkaptuk a rendszer dinamikai mátrixát, s meghatároztuk a sajátvektorokat és a megfelelő sajátértékeket (A számolás megtalálható a kurzus honlapján \q{Ising spinek dinamikája} cím alatt is).
\begin{enumerate}
    \item Ismerve az összes sajátvektort és sajátértékeket, határozzuk meg milyen valószínűséggel van a rendszer $t$ időpontban az $s_{1} = -1, s_{2} = -1$ állapotban, ha a kezdeti állapot $s_{1} = +1, s_{2} = -1$ volt.
    \item Számítsuk ki a rendszer átlagos mágnesezettségének $M \left( t \right) = \left< s_{1} + s_{2} \right>$ időfejlődését, ha a kezdeti állapotban minden konfiguráció egyenlő $\left( \frac{1}{4} \right)$ valószínűséggel van jelen.
\end{enumerate}
}

\subsection{Állapotok valószínűsége}
Az órán is látott leírás alapján keressük a következő egyensúlyi helyzetet:

\begin{equation}
    P^{\left( e \right)} \left( s_{1}, s_{2} \right)
    =
    \frac{1}{z} e^{- \beta E \left( s_{1}, s_{2} \right)}
    =
    \frac{1}{z} e^{\beta \mathcal{J} s_{1} s_{2}}
\end{equation}
Ahol $- \mathcal{J} s_{1} s_{2} \equiv \mathscr{H} = E$ a kölcsönhatás energiája, amiben $\mathcal{J}$ a rendszer egy pozitív skálafaktora. Bevezetve $\beta \mathcal{J} \equiv K$ mennyiséget, a $z$ normálási faktort a következő feltétel alapján írhatjuk fel:

\begin{equation}
    \sum_{s_{1}\ =\ \pm 1} \sum_{s_{2}\ =\ \pm 1} P^{\left( e \right)} \left( s_{1}, s_{2} \right)
    =
    \frac{1}{z} \sum_{s_{1}\ =\ \pm 1} \sum_{s_{2}\ =\ \pm 1} e^{\beta \mathcal{J} s_{1} s_{2}}
    =
    1
\end{equation}
Így kifejezhetjük $z$-t:

\begin{align}
    z
    &=
    \sum_{s_{1}\ =\ \pm 1} \sum_{s_{2}\ =\ \pm 1} e^{- \beta E \left( s_{1}, s_{2} \right)}
    \equiv
    \sum_{s_{1}\ =\ \pm 1} \sum_{s_{2}\ =\ \pm 1} e^{\beta \mathcal{J} s_{1}, s_{2}}
    =
    \sum_{s_{1}\ =\ \pm 1} \sum_{s_{2}\ =\ \pm 1} e^{K s_{1}, s_{2}}
    = \nonumber \\
    &=
    e^{K * 1 * 1} + e^{K * 1 * \left( -1 \right)} + e^{K \left( -1 \right) * 1} + e^{K \left( -1 \right) * \left( -1 \right)}
    =
    2 * \left( e^{K} + e^{-K} \right)
\end{align}
Ebből végül az adott állapot valószínűsége előáll az alábbi módon:

\begin{equation}
    P^{\left( e \right)} \left( s_{1}, s_{2} \right)
    =
    \frac{1}{z} e^{K s_{1} s_{2}}
    =
    \frac{e^{K s_{1} s_{2}}}{2 * \left( e^{K} + e^{-K} \right)}
\end{equation}