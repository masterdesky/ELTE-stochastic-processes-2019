\section{} \label{sec:2}
\quest{Vizsgáljuk az előadáson tárgyalt 2 Ising spinből álló rendszer relaxációjának problémáját! Az előadáson megkaptuk a rendszer dinamikai mátrixát, s meghatároztuk a sajátvektorokat és a megfelelő sajátértékeket (A számolás megtalálható a kurzus honlapján \q{Ising spinek dinamikája} cím alatt is).
\begin{enumerate}
    \item Ismerve az összes sajátvektort és sajátértékeket, határozzuk meg milyen valószínűséggel van a rendszer $t$ időpontban az $s_{1} = -1, s_{2} = -1$ állapotban, ha a kezdeti állapot $s_{1} = +1, s_{2} = -1$ volt.
    \item Számítsuk ki a rendszer átlagos mágnesezettségének $M \left( t \right) = \left< s_{1} + s_{2} \right>$ időfejlődését, ha a kezdeti állapotban minden konfiguráció egyenlő $\left( \frac{1}{4} \right)$ valószínűséggel van jelen.
\end{enumerate}
}

\subsection{Állapotok valószínűsége}
Az órán is látott leírás alapján keressük a következő egyensúlyi helyzetet:

\begin{equation} \label{eq:5}
    P^{\left( e \right)} \left( s_{1}, s_{2} \right)
    =
    \frac{1}{z} e^{- \beta E \left( s_{1}, s_{2} \right)}
    =
    \frac{1}{z} e^{\beta \mathcal{J} s_{1} s_{2}}
\end{equation}
Ahol $- \mathcal{J} s_{1} s_{2} \equiv \mathscr{H} = E$ a kölcsönhatás energiája, amiben $\mathcal{J}$ a rendszer egy pozitív skálafaktora. Bevezetve $\beta \mathcal{J} \equiv K$ mennyiséget, a $z$ normálási faktort a következő feltétel alapján írhatjuk fel:

\begin{equation} \label{eq:6}
    \sum_{s_{1}\ =\ \pm 1} \sum_{s_{2}\ =\ \pm 1} P^{\left( e \right)} \left( s_{1}, s_{2} \right)
    =
    \frac{1}{z} \sum_{s_{1}\ =\ \pm 1} \sum_{s_{2}\ =\ \pm 1} e^{\beta \mathcal{J} s_{1} s_{2}}
    =
    1
\end{equation}
Így kifejezhetjük $z$-t:

\begin{align} \label{eq:7}
    z
    &=
    \sum_{s_{1}\ =\ \pm 1} \sum_{s_{2}\ =\ \pm 1} e^{- \beta E \left( s_{1}, s_{2} \right)}
    \equiv
    \sum_{s_{1}\ =\ \pm 1} \sum_{s_{2}\ =\ \pm 1} e^{\beta \mathcal{J} s_{1}, s_{2}}
    =
    \sum_{s_{1}\ =\ \pm 1} \sum_{s_{2}\ =\ \pm 1} e^{K s_{1}, s_{2}}
    = \nonumber \\
    &=
    e^{K * 1 * 1} + e^{K * 1 * \left( -1 \right)} + e^{K \left( -1 \right) * 1} + e^{K \left( -1 \right) * \left( -1 \right)}
    =
    2 * \left( e^{K} + e^{-K} \right)
\end{align}
Ebből végül az adott állapot valószínűsége előáll az alábbi módon:

\begin{equation} \label{eq:8}
    P^{\left( e \right)} \left( s_{1}, s_{2} \right)
    =
    \frac{1}{z} e^{K s_{1} s_{2}}
    =
    \frac{e^{K s_{1} s_{2}}}{2 * \left( e^{K} + e^{-K} \right)}
\end{equation}
Ebből látható, hogy a rendszer energiája akkor alacsonyabb, ha a spinek azonos irányba állnak, míg magasabb, ha azok ellentétesek egymással:

\begin{equation} \label{eq:9}
    P^{\left( e \right)} \left( \uparrow \uparrow \right)
    =
    P^{\left( e \right)} \left( \downarrow \downarrow \right)
    =
    \frac{e^{K}}{z}
\end{equation}
\begin{equation} \label{eq:10}
    P^{\left( e \right)} \left( \uparrow \downarrow \right)
    =
    P^{\left( e \right)} \left( \downarrow \uparrow \right)
    =
    \frac{e^{-K}}{z}
\end{equation}
Bevezetve egy $w_{1} \left( s_{1}, s_{2} \right)$ és egy $w_{2} \left( s_{1}, s_{2} \right)$ rátát, mellyel a két spin állapotváltozását jellemezzük, felírható a rendszer időfejlődését kifejező Master-egyenlet:

\begin{align} \label{eq:11}
    \frac{\partial P \left( s_{1}, s_{2}, t \right)}{\partial t}
    &=
    - \left[ w_{1} \left( s_{1}, s_{2} \right) + w_{2} \left( s_{1}, s_{2} \right) \right] P \left( s_{1}, s_{2}, t \right)
    + \nonumber \\
    &+
    w_{1} \left( - s_{1}, s_{2} \right) P \left( - s_{1}, s_{2}, t \right) + w_{2} \left( s_{1}, - s_{2} \right) P \left( s_{1}, - s_{2}, t \right)
\end{align}
Feltételezve, hogy a rendszer az egyensúly felé tart, a Master-egyenlet egy stacionárius megoldása ismert módon felírható a részletes egyensúly elve alapján:

\begin{equation} \label{eq:12}
    w_{1} \left( s_{1}, s_{2} \right) P^{\left( e \right)} \left( s_{1}, s_{2} \right)
    =
    w_{1} \left( - s_{1}, s_{2} \right) P^{\left( e \right)} \left( - s_{1}, s_{2} \right)
\end{equation}
Ezzel megkaptuk az $w_{1}$ átmenetek arányát:

\begin{equation} \label{eq:13}
    \frac{w_{1} \left( s_{1}, s_{2} \right)}{w_{1} \left( - s_{1}, s_{2} \right)}
    =
    \frac{P^{\left( e \right)} \left( - s_{1}, s_{2} \right)}{P^{\left( e \right)} \left( s_{1}, s_{2} \right)}
    =
    \frac{e^{- K s_{1} s_{2}}}{e^{K s_{1} s_{2}}}
    =
    e^{- 2 K s_{1} s_{2}}
\end{equation}
Analóg módon a $w_{2}$-re is felírhatjuk a fentieket:

\begin{equation} \label{eq:14}
    w_{2} \left( s_{1}, s_{2} \right) P^{\left( e \right)} \left( s_{1}, s_{2} \right)
    =
    w_{2} \left( s_{1}, - s_{2} \right) P^{\left( e \right)} \left( s_{1}, - s_{2} \right)
\end{equation}
Ebből pedig ugyanúgy megkaphatjuk a $w_{2}$ ráták arányát:

\begin{equation} \label{eq:15}
    \frac{w_{2} \left( s_{1}, s_{2} \right)}{w_{2} \left( s_{1}, - s_{2} \right)}
    =
    \frac{P^{\left( e \right)} \left( s_{1}, - s_{2} \right)}{P^{\left( e \right)} \left( s_{1}, s_{2} \right)}
    =
    \frac{e^{- K s_{1} s_{2}}}{e^{K s_{1} s_{2}}}
    =
    e^{- 2 K s_{1} s_{2}}
\end{equation}
Ezen fentiek felhasználásával beláthatjuk, hogy az alábbi spin-kicserélődési ráták kielégítik a (\ref{eq:13}) és (\ref{eq:15}) egyenleteket:

\begin{equation} \label{eq:16}
    w_{1} \left( \uparrow \uparrow \right)
    =
    w_{2} \left( \uparrow \uparrow \right)
    =
    w_{1} \left( \downarrow \downarrow \right)
    =
    w_{2} \left( \downarrow \downarrow \right)
    =
    e^{-2 K}
\end{equation}
\begin{equation} \label{eq:17}
    w_{1} \left( \uparrow \downarrow \right)
    =
    w_{2} \left( \uparrow \downarrow \right)
    =
    w_{1} \left( \downarrow \uparrow \right)
    =
    w_{2} \left( \downarrow \uparrow \right)
    =
    1
\end{equation}
Ezeknek a felhasználásával a Master-egyenlet az egyes esetekre felírható a következő formában:

\begin{equation} \label{eq:18}
    \frac{\partial P \left( \uparrow \uparrow, t \right)}{\partial t}
    =
    -2 e^{-2 K} P \left( \uparrow \uparrow, t \right)
    +
    1 * P \left( \downarrow \uparrow, t \right)
    +
    1 * P \left( \uparrow \downarrow, t \right)
    +
    0 * P \left( \downarrow \downarrow, t \right)
\end{equation}
\begin{equation} \label{eq:19}
    \frac{\partial P \left( \downarrow \uparrow, t \right)}{\partial t}
    =
    e^{-2 K} * P \left( \uparrow \uparrow, t \right)
    -
    2 * P \left( \downarrow \uparrow, t \right)
    +
    0 * P \left( \uparrow \downarrow, t \right)
    +
    e^{-2 K} * P \left( \downarrow \downarrow, t \right)
\end{equation}
\begin{equation} \label{eq:20}
    \frac{\partial P \left( \uparrow \downarrow, t \right)}{\partial t}
    =
    e^{-2 K} * P \left( \uparrow \uparrow, t \right)
    +
    0 * P \left( \downarrow \uparrow, t \right)
    -
    2 * P \left( \uparrow \downarrow, t \right)
    +
    e^{-2 K} * P \left( \downarrow \downarrow, t \right)
\end{equation}
\begin{equation} \label{eq:21}
    \frac{\partial P \left( \downarrow \downarrow, t \right)}{\partial t}
    =
    0 * P \left( \uparrow \uparrow, t \right)
    +
    1 * P \left( \downarrow \uparrow, t \right)
    +
    1 * P \left( \uparrow \downarrow, t \right)
    -
    2 e^{-2 K} * P \left( \downarrow \downarrow, t \right)
\end{equation}
Ezeket a fenti egyenleteket egy $4 \times 4$ mátrix és egy $4$ elemű vektor szorzataként is megfogalamzhatjuk. Vezessük be az alábbi vektort (használva az alábbi papírban található jelölést: \cite{heatbath}):

\begin{equation} \label{eq:22}
    \vec{P} \left( t \right)
    =
    \begin{pmatrix}
        P \left( \uparrow \uparrow, t \right) \\
        P \left( \downarrow \uparrow, t \right) \\
        P \left( \uparrow \downarrow, t \right) \\
        P \left( \downarrow \downarrow, t \right)
    \end{pmatrix}
\end{equation}
Valamint az alábbi mátrixot:

\begin{equation} \label{eq:23}
    \mathcal{A}
    =
    \begin{pmatrix}
        -2 e^{-2 K}  &  1   &  1   &  0            \\
        e^{-2 K}     &  -2  &  0   &  e^{-2 K}     \\
        e^{-2 K}     &  0   &  -2  &  e^{-2 K}     \\
        0            &  1   &  0   &  -2 e^{-2 K}  \\
    \end{pmatrix}
\end{equation}
Így a Master-egyenlet a következő formára redukálódik:

\begin{equation} \label{eq:24}
    \partial_{t} \vec{P} \left( t \right)
    =
    \mathcal{A} \vec{P} \left( t \right)
\end{equation}
Az egyensúlyi állapotot leíró (\ref{eq:8}) stacionárius eset pedig a következő alakot ölti:

\begin{equation} \label{eq:25}
    \vec{P}^{\left( e \right)}
    =
    \vec{P}^{\left( 1 \right)}
    =
    \frac{1}{2 * \left( e^{K} + e^{-K} \right)}
    \begin{pmatrix}
        e^{K} \\
        e^{-K} \\
        e^{-K} \\
        e^{K}
    \end{pmatrix}
\end{equation}
Ez az $\mathcal{A}$ mátrix, $\lambda_{1} = 0$ sajátértékhez tartozó sajátvektora. A maradék sajátértéket és sajátvektort kiszámítva az órán látottak alapján már információval tudunk szolgálni az egyes állapotok időfejlődéséről. A sajátvektorok és sajátértékek a követekezők:

\begin{equation} \label{eq:26}
    \lambda_{2}
    =
    -2 * \left(e^{-2K} + 1 \right)
    \quad \quad \quad
    \vec{P}^{\left( 2 \right)}
    =
    \begin{pmatrix}
        1 \\
        -1 \\
        -1 \\
        1
    \end{pmatrix}
\end{equation}
\begin{equation} \label{eq:27}
    \lambda_{3}
    =
    -2 * e^{-2K}
    \quad \quad \quad
    \vec{P}^{\left( 3 \right)}
    =
    \begin{pmatrix}
        1 \\
        0 \\
        0 \\
        -1
    \end{pmatrix}
\end{equation}
\begin{equation} \label{eq:28}
    \lambda_{4}
    =
    -2
    \quad \quad \quad
    \vec{P}^{\left( 4 \right)}
    =
    \begin{pmatrix}
        0 \\
        1 \\
        -1 \\
        0
    \end{pmatrix}
\end{equation}
Az egyes állapotok időfejlődése a fentiek segítségével megadható a következő módon:

\begin{equation} \label{eq:29}
    \vec{P}^{\left ( i \right)} \left ( t \right)
    =
    a_{i} * e^{\lambda_{i} t} \vec{P}^{\left ( i \right)}
\end{equation}
Ennek segítségével felírhatjuk az általános megoldás időfejlődését:

\begin{equation} \label{eq:30}
    \vec{P} \left ( t \right)
    =
    \vec{P}^{\left ( e \right)}
    +
    \sum_{i\ =\ 2}^{4} a_{i} * e^{\lambda_{i} t} \vec{P}^{\left ( i \right)}
\end{equation}
Melyekben az $a_{i}$ együtthatók meghatározhatóak az alábbi kezdőfeltételből:

\begin{equation} \label{eq:31}
    \vec{P} \left(0 \right)
    =
    \vec{P}^{\left ( e \right)}
    +
    \sum_{i\ =\ 2}^{4} a_{i} \vec{P}^{i}
\end{equation}
A feladat első részében meghatározandó a $P \left( \downarrow \downarrow, t \right)$, ha a kezdeti állatot $P \left( \uparrow \downarrow, 0 \right)$. Vektoros formában a kezdeti- és végállapot a következőképp írható fel:

\begin{equation} \label{eq:32}
    \vec{P} \left( t = 0 \right)
    =
    \begin{pmatrix}
        0 \\
        0 \\
        1 \\
        0
    \end{pmatrix}
    \quad \to \quad
    \vec{P} \left( t \right)
    =
    \begin{pmatrix}
        0 \\
        0 \\
        0 \\
        1
    \end{pmatrix}
\end{equation}
A (\ref{eq:31}) egyenletben leírtak alapján ki tudjuk fejezni az egyes komponenseket, felhasználva a (\ref{eq:25}) - (\ref{eq:28}) mennyiségeket:

\begin{equation} \label{eq:33}
    0
    =
    \frac{e^{K}}{z} a_{1} + 1 * a_{2} + 1 * a_{3} + 0 * a_{4}
    =
    \frac{e^{K}}{z} a_{1} + 1 * a_{2} + 1 * a_{3}
\end{equation}
\begin{equation} \label{eq:34}
    0
    =
    \frac{e^{-K}}{z} a_{1} - 1 * a_{2} + 0 * a_{3} + 1 * a_{4}
    =
    \frac{e^{-K}}{z} a_{1} - 1 * a_{2} + 1 * a_{4}
\end{equation}
\begin{equation} \label{eq:35}
    1
    =
    \frac{e^{-K}}{z} a_{1} - 1 * a_{2} + 0 * a_{3} - 1 * a_{4}
    =
    \frac{e^{-K}}{z} a_{1} - 1 * a_{2} - 1 * a_{4}
\end{equation}
\begin{equation} \label{eq:36}
    0
    =
    \frac{e^{K}}{z} a_{1} + 1 * a_{2} - 1 * a_{3} + 0 * a_{4}
    =
    \frac{e^{K}}{z} a_{1} + 1 * a_{2} - 1 * a_{3}
\end{equation}
Kivonva (\ref{eq:33})-ból (\ref{eq:36})-ot, kapjuk a következőt:

\begin{equation} \label{eq:37}
    0 - 0
    =
    \frac{e^{K}}{z} a_{1} - \frac{e^{K}}{z} a_{1} + a_{2} - a_{2} + a_{3} - \left( - a_{3} \right)
    \quad \to \quad
    \boxed{0 = \cancel{2} * a_{3}}
\end{equation}
Euztán szintén kivonva (\ref{eq:34})-ből (\ref{eq:35})-öt, kapjuk a következőt:
\begin{equation} \label{eq:38}
    0 - 1
    =
    \frac{e^{-K}}{z} a_{1} - \frac{e^{-K}}{z} a_{1} - a_{2} - \left(- a_{2} \right) + a_{4} - \left( - a_{4} \right)
    \quad \to \quad
    \boxed{-\frac{1}{2} = a_{4}}
\end{equation}
Összeadva a (\ref{eq:33}) és (\ref{eq:34}) egyenleteket, felhasználva a fent kapott $a_{2}$ értékét, megkaphatjuk $a_{1}$-et:

\begin{equation} \label{eq:39}
    0 + 0
    =
    \frac{e^{K}}{z} a_{1} + \frac{e^{-K}}{z} a_{1} + a_{2} - a_{2} + 0 - \frac{1}{2}
    \quad \to \quad
    \frac{z}{2 \left( e^{K} + e^{-K} \right)}
    =
    a_{1}
\end{equation}
A (\ref{eq:7}) alapján azonban $z$ értékét behelyettesítve:

\begin{equation} \label{eq:40}
    \frac{z}{2 \left( e^{K} + e^{-K} \right)}
    =
    \frac{2 \left( e^{K} + e^{-K} \right)}{2 \left( e^{K} + e^{-K} \right)}
    =
    \boxed{
    1
    =
    a_{1}}
\end{equation}
Míg az $a_{2}$ értéke egy tetszőleges, $a_{2}$-t tartalmazó egyenlet alapján meghatározható, legyen ez pl. a (\ref{eq:36})-os:

\begin{equation} \label{eq:41}
    0
    =
    \frac{e^{K}}{z} * 1 + 1 * a_{2} - 1 * 0
    \quad \to \quad
    -\frac{e^{K}}{z}
    =
    \boxed{
    -\frac{e^{K}}{2 \left( e^{K} + e^{-K} \right)}
    =
    a_{2}}
\end{equation}
Ezek segítségével a (\ref{eq:30}) és (\ref{eq:32}) alapján a $P \left( \downarrow \downarrow, t \right)$ a következő lesz:

\begin{align} \label{eq:42}
    P \left( \downarrow \downarrow, t \right)
    &=
    \frac{e^{K}}{z} a_{1} * e^{\lambda_1 t} + 1 * a_{2} * e^{\lambda_2 t} - 1 * a_{3} * e^{\lambda_3 t}
    = \nonumber \\
    &=
    \frac{e^{K}}{2 \left( e^{K} + e^{-K} \right)} * e^{0 * t}
    -
    \frac{e^{K}}{2 \left( e^{K} + e^{-K} \right)} * e^{-2 * \left( e^{-2 K} + 1 \right) t}
    = \nonumber \\
    &=
    \boxed{
    \left( 1 - e^{-2 * \left( e^{-2 K} + 1 \right) t} \right) * \frac{e^{K}}{2 \left( e^{K} + e^{-K} \right)}
    }
\end{align}

\subsection{A mágnesezettség időfejlődése}
A mágnesezettség időfüggése leírható az egyes állapotokra történő összegzéssel, magának a mágnesezettség $m\left( t \right) = \left< s_{1} + s_{2} \right>$ definíciója segítségével:

\begin{equation} \label{eq:43}
    m \left( t \right)
    =
    \sum_{s_{1}\ =\ \pm 1} \sum_{s_{2}\ =\ \pm 1} \left( s_{1} + s_{2} \right)P(s_{1}, s_{2}, t)
\end{equation}
A feladat szövegében szerepel, hogy $\vec{P}$ a következő alakú kezdetben:

\begin{equation} \label{eq:44}
    \vec{P} \left( t = 0 \right)
    =
    \begin{pmatrix}
        1/4 \\
        1/4 \\
        1/4 \\
        1/4
    \end{pmatrix}
\end{equation}
A (\ref{eq:31}) egyenletben leírtak alapján ki tudjuk fejezni az egyes komponenseket, felhasználva a (\ref{eq:25}) - (\ref{eq:28}) mennyiségeket:

\begin{equation} \label{eq:45}
    \frac{1}{4}
    =
    \frac{e^{K}}{z} a_{1} + 1 * a_{2} + 1 * a_{3} + 0 * a_{4}
    =
    \frac{e^{K}}{z} a_{1} + 1 * a_{2} + 1 * a_{3}
\end{equation}
\begin{equation} \label{eq:46}
    \frac{1}{4}
    =
    \frac{e^{-K}}{z} a_{1} - 1 * a_{2} + 0 * a_{3} + 1 * a_{4}
    =
    \frac{e^{-K}}{z} a_{1} - 1 * a_{2} + 1 * a_{4}
\end{equation}
\begin{equation} \label{eq:47}
    \frac{1}{4}
    =
    \frac{e^{-K}}{z} a_{1} - 1 * a_{2} + 0 * a_{3} - 1 * a_{4}
    =
    \frac{e^{-K}}{z} a_{1} - 1 * a_{2} - 1 * a_{4}
\end{equation}
\begin{equation} \label{eq:48}
    \frac{1}{4}
    =
    \frac{e^{K}}{z} a_{1} + 1 * a_{2} - 1 * a_{3} + 0 * a_{4}
    =
    \frac{e^{K}}{z} a_{1} + 1 * a_{2} - 1 * a_{3}
\end{equation}
Az előző feladtban látotakhoz hasonló módokon járhatunk el. Vonjuk ki a (\ref{eq:48})-ból (\ref{eq:45})-öt. Ekkor megkapjuk $a_{3}$ értékét:

\begin{equation} \label{eq:49}
    \frac{1}{4} - \frac{1}{4}
    =
    \frac{e^{K}}{z} a_{1} + 1 * a_{2} - 1 * a_{3} - \frac{e^{K}}{Z} a_{1} - 1 * a_{2} - 1 * a_{3}
    \quad \to \quad
    \boxed{0 = a_{3}}
\end{equation}
Ezután (\ref{eq:47})-ből vonjuk ki (\ref{eq:46})-ot:

\begin{equation}  \label{eq:50}
    \frac{1}{4} - \frac{1}{4}
    =
    \frac{e^{-K}}{z} a_{1} - 1 * a_{2} - 1 * a_{4} - \frac{e^{-K}}{z} a_{1} + 1 * a_{2} - 1 * a_{4}
    \quad \to \quad
    \boxed{0 = a_{4}}
\end{equation}
Most adjuk össze (\ref{eq:45})-öt és (\ref{eq:46})-ot:

\begin{equation}  \label{eq:51}
    \frac{1}{4} + \frac{1}{4}
    =
    \frac{e^{K}}{z} a_{1} + 1 * a_{2} + 1 * a_{3} + \frac{e^{-K}}{z} a_{1} - 1 * a_{2} + 1 * a_{4}
    =
    \frac{e^{K}}{z} a_{1} + \frac{e^{-K}}{Z} a_{1}
\end{equation}
Így $a_{1}$ értékére a következő kifejezést kapjuk:

\begin{equation} \label{eq:52}
    \frac{1}{2}
    =
    \left( \frac{e^{K}}{z} + \frac{e^{-K}}{Z} \right) a_{1}
    \quad \to \quad
    \frac{z}{2 * \left( e^{K} + e^{-K} \right)}
    =
    \frac{2 * \left( e^{K} + e^{-K} \right)}{2 * \left( e^{K} + e^{-K} \right)}
    =
    \boxed{
    1
    =
    a_{1}}
\end{equation}
Végül pedig $a_{2}$ értékét határozzuk meg pl. a (\ref{eq:45}) egyeneltből:

\begin{equation} \label{eq:53}
    \frac{1}{4}
    =
    \frac{e^{K}}{z} a_{1} + 1 * a_{2} + 1 * a_{3}
    =
    \frac{e^{K}}{z} + a_{2}
    \quad \to \quad
    \boxed{
    \frac{1}{4} - \frac{e^{K}}{z}
    =
    a_{2}}
\end{equation}
Ezen kapott értékek felhasználásával megkonstruálhatjuk a $\vec{P}$ vektort:

\begin{equation} \label{eq:54}
    \vec{P}
    =
    \begin{pmatrix}
        \frac{e^{K}}{z}  * e^{\lambda_{1} t} +  \left( \frac{1}{4} - \frac{e^{K}}{z} \right) * e^{\lambda_2 t} \\
        \frac{e^{-K}}{z} * e^{\lambda_{1} t} -  \left( \frac{1}{4} + \frac{e^{K}}{z} \right) * e^{\lambda_2 t} \\
        \frac{e^{-K}}{z} * e^{\lambda_{1} t} -  \left( \frac{1}{4} + \frac{e^{K}}{z} \right) * e^{\lambda_2 t} \\
        \frac{e^{K}}{z}  * e^{\lambda_{1} t} +  \left( \frac{1}{4} - \frac{e^{K}}{z} \right) * e^{\lambda_2 t}
    \end{pmatrix}
    =
    \begin{pmatrix}
        \frac{e^{K}}{z}   +  \left( \frac{1}{4} - \frac{e^{K}}{z} \right) * e^{-2 * \left(e^{-2K} + 1 \right) t} \\
        \frac{e^{-K}}{z}  -  \left( \frac{1}{4} + \frac{e^{K}}{z} \right) * e^{-2 * \left(e^{-2K} + 1 \right) t} \\
        \frac{e^{-K}}{z}  -  \left( \frac{1}{4} + \frac{e^{K}}{z} \right) * e^{-2 * \left(e^{-2K} + 1 \right) t} \\
        \frac{e^{K}}{z}   +  \left( \frac{1}{4} - \frac{e^{K}}{z} \right) * e^{-2 * \left(e^{-2K} + 1 \right) t}
    \end{pmatrix}
\end{equation}
Az (\ref{eq:43})-as összefüggés alapján most már felírhatjuk a keresett $m \left( t \right)$ mágnesezettséget:

\begin{align} \label{eq:55}
    m \left( t \right)
    &=
    \sum_{s_{1}\ =\ \pm 1} \sum_{s_{2}\ =\ \pm 1} \left( s_{1} + s_{2} \right)P(s_{1}, s_{2}, t)
    = \nonumber \\
    &=
    \left( 1 + 1 \right) * \left( \frac{e^{K}}{z} + \left( \frac{1}{4} - \frac{e^{K}}{z} \right) * e^{-2 * \left(e^{-2K} + 1 \right) t} \right)
    +
    \overbrace{\left( 1 - 1 \right) * \vec{P}^{\left( 2 \right)}}^{=\ 0}
    +
    \overbrace{\left( -1 + 1 \right) * \vec{P}^{\left( 3 \right)}}^{=\ 0}
    + \nonumber \\
    &+
    \left( -1 + -1 \right) * \left( \frac{e^{K}}{z} + \left( \frac{1}{4} - \frac{e^{K}}{z} \right) * e^{-2 * \left(e^{-2K} + 1 \right) t} \right)
\end{align}
Ebben ahogy jelöltem is, csak a $\vec{P}^{1} = P \left( \uparrow \uparrow \right)$ és $\vec{P}^{4} = P \left( \downarrow \downarrow \right)$ spinállapotok adnak járulékot. Tovább oldva az egyenletet:

\begin{align} \label{eq:56}
    m \left( t \right)
    &=
    2 * \left( \frac{e^{K}}{z} + \left( \frac{1}{4} - \frac{e^{K}}{z} \right) * e^{-2 * \left(e^{-2K} + 1 \right) t} \right)
    -
    2 * \left( \frac{e^{K}}{z} + \left( \frac{1}{4} - \frac{e^{K}}{z} \right) * e^{-2 * \left(e^{-2K} + 1 \right) t} \right)
    = \nonumber \\
    &=
    2 * \left[
    \underbrace{\left( \frac{e^{K}}{z} + \left( \frac{1}{4} - \frac{e^{K}}{z} \right) * e^{-2 * \left(e^{-2K} + 1 \right) t} \right)
    -
    \left( \frac{e^{K}}{z} + \left( \frac{1}{4} - \frac{e^{K}}{z} \right) * e^{-2 * \left(e^{-2K} + 1 \right) t} \right)}_{=\ 0}
    \right]
\end{align}
Tehát általánosan feltételezhetjük, hogy az azonos valószínűségű konfigurációkkal induló rendszerben:

\begin{equation} \label{eq:57}
    \boxed{
    m \left( t \right)
    =
    2 * \left[ P \left( \uparrow \uparrow, t \right) - P \left( \downarrow \downarrow, t \right) \right]
    =
    0
    }
\end{equation}
Ha ez nem csak elszámolás, akkor ennek egy lehetséges magyarázatát könnyű megadni. A $\left( \uparrow \downarrow \right)$ és $\left( \downarrow \uparrow \right)$ spinállapotok random mozgás hatására mindig a kisebb energiájú $\left( \uparrow \uparrow \right)$ és $\left( \downarrow \downarrow \right)$ állapotok felé mozognak. Mivel a rendszer teljesen szimmetrikus, így mind a $\left( \uparrow \uparrow \right)$ és $\left( \downarrow \downarrow \right)$ állapotok számának növekedési rátája megegyezik. Mivel a mágnesezettséget ezen kettő utóbbi határozza meg a (\ref{eq:57}) összefüggésnek megfelelően, így érthető, hogy ez a mennyiség $0$ lesz, ugyanis állandóan azonos számú $\left( \uparrow \uparrow \right)$ és $\left( \downarrow \downarrow \right)$ lesz a rendszerben.