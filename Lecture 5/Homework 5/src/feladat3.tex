\section{} \label{sec:3}
\quest{Meredek hegyoldalban függőlegesen $l$ távolságra vannak a kapaszkodók. A hegymászó $w$ rátával lép felfelé, s $w_{0}$ annak a rátája, hogy lecsúszik egy szintet, s onnan folytatja a mászást. \\
Feladatok:
\begin{enumerate}
    \item Írjuk fel az egyenletet, amely meghatározza, hogy a hegymászó milyen $P_{n}$ valószínűséggel van $nl$ magasságban!
    \item Használjuk a generátorfüggvény formalizmust a stacionárius eloszlás kiszámítására!
    \item Határozzuk meg, hogy átlagosan milyen magasra jut a hegymászó!
    \item Van itt a hasonlóság a sorbanállás problémájával?
\end{enumerate}
}