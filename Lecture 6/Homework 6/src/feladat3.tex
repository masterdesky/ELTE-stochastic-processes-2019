\section{} \label{sec:3}
K: \textit{Legyen egy sztochasztikus változó, $x$, momentum-generátor függvénye $G \left( s \right)$. A normalizációból következik, hogy $G  \left( 0 \right) = 1$, s tudjuk, hogy az $x$ momentumai $G$ deriváltjain keresztül kifejezhetők:}

\begin{equation*}
    \left< x \right>
    =
    - \left. \frac{d G \left( s \right)}{d s} \right\rvert_{s = 0}
    \quad \text{,} \dots \text{,} \quad
    \left< x^{k} \right>
    =
    \left( -1 \right)^{k} \left. \frac{d^{k} G \left( s \right)}{d s^{k}} \right\rvert_{s = 0}
\end{equation*}
\textit{A kumuláns generátor függvény a momentum-generátor függvény logaritmusa, $\Phi \left( s \right) = \ln \left( G \left( s \right) \right)$, s a kumulánsokat a következőképpen kapjuk:}

\begin{equation*}
    \left< \kappa_{1} \right>
    =
    - \left. \frac{d \Phi \left( s \right)}{d s} \right\rvert_{s = 0}
    \quad \text{,} \dots \text{,} \quad
    \left< \kappa_{k} \right>
    =
    \left( -1 \right)^{k} \left. \frac{d^{k} \Phi \left( s \right)}{d s^{k}} \right\rvert_{s = 0}
\end{equation*}
\textit{Az első kumulánsokat könnyű kiszámolni, s egyszerű értelmük van:}

\begin{equation*}
    \left< \kappa_{1} \right>
    =
    \left< x \right>
    \quad \text{,} \quad
    \left< \kappa_{2} \right>
    =
    \left< x^{2} \right> - \left< x \right>^{2}
\end{equation*}
\textit{Feladat: Számítsuk ki a 3. kumulánst $\left( \kappa_{3} \right)$ a momentumokon keresztül! Mi lesz $\kappa_{3}$ értéke, ha $x$ eloszlásfüggvénye szimmetrikus $\left[ P \left( -x \right) = P \left( x \right) \right]$?}

\subsection{A 3. kumuláns kiszámítása}

A harmadik kumuláns definícióját a feladat szövegében is leírt általános definíció segítségével fogalamzhatjuk:

\begin{align}
    \kappa_{3}
    &=
    \left( -1 \right)^{3} \left. \frac{d^{3} \Phi \left( s \right)}{d s^{3}} \right\rvert_{s = 0}
    =
    - \left. \frac{d^{3} \Phi \left( s \right)}{d s^{3}} \right\rvert_{s = 0}
    =
    - \left. \frac{d^{3} \ln \left( G \left( s \right) \right)}{d s^{3}} \right\rvert_{s = 0}
    = \nonumber \\
    &=
    - \left. \frac{d^{2}}{d s^{2}}
    \left(
    \frac{1}{G \left( s \right)} \frac{d G \left( s \right)}{d s}
    \right) \right\rvert_{s = 0}
    =
    - \left. \frac{d}{d s}
    \left[
    \frac{1}{G \left( s \right)} \frac{d^{2} G \left( s \right)}{d s^{2}}
    -
    \frac{1}{G^{2} \left( s \right)} \left( \frac{d G \left( s \right)}{d s} \right)^{2}
    \right] \right\rvert_{s = 0}
    = \nonumber \\
    &=
    - \left[
    \frac{1}{G \left( s \right)} \frac{d^{3} G \left( s \right)}{d s^{3}}
    -
    \frac{1}{G^{2} \left( s \right)} \frac{d G \left( s \right)}{d s} \frac{d^{2} G \left( s \right)}{d s^{2}}
    \right.
    + \nonumber \\
    &+
    \left. \left.
    2 \frac{1}{G^{3} \left( s \right)} \left( \frac{d G \left( s \right)}{d s} \right)^{3}
    -
    2 \frac{1}{G^{2} \left( s \right)} \frac{d G \left( s \right)}{d s} \frac{d^{2} G \left( s \right)}{d s^{2}}
    \right] \right\rvert_{s = 0}
    = \nonumber \\
    &=
    \underbrace{-
    \left. \frac{1}{G \left( s \right)} \frac{d^{3} G \left( s \right)}{d s^{3}}\right\rvert_{s = 0}
    }_{\left< x^{3} \right>}
    \underbrace{
    +
    \left. 3 \frac{1}{G^{2} \left( s \right)} \frac{d G \left( s \right)}{d s} \frac{d^{2} G \left( s \right)}{d s^{2}} \right\rvert_{s = 0}
    }_{- 3 * \left< x^{2} \right> * \left< x \right>}
    \underbrace{
    -
    \left. 2 \frac{1}{G^{3} \left( s \right)} \left( \frac{d G \left( s \right)}{d s} \right)^{3} \right\rvert_{s = 0}
    }_{2 * \left< x \right>^{3}}
    = \nonumber \\
    &=
    \boxed{\left< x^{3} \right> - 3 * \left< x^{2} \right> * \left< x \right> + 2 * \left< x \right>^{3}}
\end{align}
Mely eredmény valóban a harmadik kumuláns értéke.
\\ \\
Ha a kérdéses $x$ valószínűségi változó eloszlásfüggvénye azonban szimmetrikus, ennek értéke $0$, ahogy minden másik, páratlan rendű kumuláns értéke is. Ez megindokolható a várható értékek elemzésével. Ha egy $x$ valószínűségi változó eloszlása szimmetrikus, az azt jelenti, hogy $x$ és $-x$ eloszlása megegyezik, tehát minden $f$ függvényre:

\begin{equation}
    \left< f \left( x \right) \right>
    =
    \left< f \left( -x \right) \right>
\end{equation}
Bővítsük ki ezt arra az esetre, amikor $f \left( x \right) = x^{n}$:

\begin{equation}
    \left< \left( x \right)^{n} \right>
    =
    \left< \left( -x \right)^{n} \right>
    =
    \left< \left( -1 \right)^{n} \left( x \right)^{n} \right>
\end{equation}
Minden esetben, amikor $n$ páratlan, akkor ez a következő alakot ölti:

\begin{equation}
    \left< \left( x \right)^{n} \right>
    =
    \left< - \left( x \right)^{n} \right>
    =
    -\left< \left( x \right)^{n} \right>
\end{equation}
Ami kizárólag akkor lehetséges, ha

\begin{equation}
    \boxed{\left< \left( x \right)^{n} \right>
    =
    0}
    \quad \quad
    \text{ha $n$ páratlan}
\end{equation}