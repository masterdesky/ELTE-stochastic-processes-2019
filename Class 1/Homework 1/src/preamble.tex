\usepackage[utf8]{inputenc}
\usepackage[english,magyar]{babel}
\usepackage[T1]{fontenc}

\usepackage{fullpage}
\usepackage{amsmath,amsthm,amsfonts,amssymb,amscd}
\usepackage{lastpage}
\usepackage{fancyhdr}
\usepackage{mathrsfs}
\usepackage{xcolor}
\usepackage{graphicx}
\usepackage{cancel}

% Hyperlinks
\usepackage[unicode]{hyperref}
\hypersetup{
    colorlinks=true,
    linkcolor=blue,
    filecolor=magenta,      
    urlcolor=cyan,
    citecolor = cyan,
    pdftitle={Stochastic Processes - 1.},
    pdfpagemode=FullScreen,
}

% Section titles
\usepackage{titlesec}
\titleformat{\section}{\bfseries\large}{\MakeUppercase{\thesection. Feladat}}{0em}{}
\titleformat{\subsection}{\bfseries\small}{\thesubsection. }{0em}{}
 
% Quotes
\usepackage[autostyle=false]{csquotes}
\newcommand{\q}[1]{„#1''} % Redefine quotations

% \cdot instead of asterisk (*) symbol
\mathcode`\*="8000
{\catcode`\*\active\gdef*{\cdot}}

% Double underline
\def\doubleunderline#1{\underline{\underline{#1}}}

% Question
\def\quest#1{K: \emph{#1}}

% Headers/footers
\setlength{\parindent}{0.0in}
\setlength{\parskip}{0.05in}

\newcommand\course{Véletlen fizikai folyamatok}
\newcommand\hwnumber{1}
\newcommand\Name{Pál Balázs}
\newcommand\Neptun{UXB26I}

\pagestyle{fancyplain}
\headheight 35pt
\lhead{\Name\\\Neptun} 
\chead{\textbf{\Large Véletlen fizikai folyamatok\\\hwnumber. házifeladat}}
\rhead{2019. február 19.}
\lfoot{}
\cfoot{\small\thepage}
\rfoot{}
\headsep 1.5em

% Biblography
\usepackage[defernumbers=true,backend=biber,sorting=none]{biblatex}
\addbibresource{bibliography.bib}