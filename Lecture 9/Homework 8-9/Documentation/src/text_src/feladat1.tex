\section{} \label{sec:1}
K:\ Szimuláljuk az órán tárgyalt véletlen rekurzív fát (minden lépésben egy új csúcsot adunk a hálózathoz, s az új csúcsot egyenlő valószínűséggel kötjük a meglevő csúcsok egyikéhez).
\\ \\
Feladatok:

\begin{enumerate}[label=\roman*]
    \item Határozzuk meg a csúcsok fokszámeloszlását $P_{k} = N_{k}/N$-t, ahol N a csúcsok száma, $N_{k}$ pedig a $k$ éllel rendelkező csúcsok száma.
    \item Vizsgájuk mekkora $N$ kell ahhoz, hogy az eloszlásfüggvény $P_{k}$ hibája kisebb legyen mint $10\%$ minden $k \leq 10$-re. Figyelem, az egzakt eloszlásfüggvényt az órán kiszámoltuk!
    \item Határozzuk meg az átlagos fokszámot mind elméletileg, mind pedig a szimulációkból!
    \item Találjuk meg a maximális fokszámú csúcsot a fenti szimulációkban generált hálózatokban. Többször megismételve a szimulációkat $N = 100$, $1000$ és $10000$ esetére, határozzuk meg a maximális fokszám átlagát, $\left< k_{max} \right>$-t! Látunk trendet az eredményekben?
\end{enumerate}