\section*{\bfseries\large\MakeUppercase{Megoldás}}

\subsection{Megvalósítás}
A feladatokat megvalósító programok forráskódjának mindegyikét egy Jupyter Notebook-ban futó Python 3.7 kernel alatt valósítottam meg, valamint az ábrákat is ebben készítettem. Ezeket GitHub-on mind elérhetővé is tettem\cite{github}.
\\ \\
Az első feladat során a \rrtt kellett szimuláljuk, tetszőleges $N$ csúcs/élszám darabszámra. Ennek szimulálása a következő lépések elvégzéséből tevődik össze:

\begin{enumerate}
    \item Kezdetben lehelyezünk egyetlen pontot az üres térbe. Triviálisan mivel ő az egyetlen, így fokszáma $0$, nem csatlakozhat semmilyen másik ponthoz (hisz eleve nincs másik pont).
    \item Első lépésben lehelyezünk még egy pontot a térbe, és véletlenszerűen - azonos valószínűséggel - kiválasztunk egy már meglévőt (jelen esetben ebből egyetlen egy darab van), melyet egy éllel összekötünk az újonnan behozott csúccsal. Minden lehetséges esetben az első két pont szükségszerűen egymással lesz összekötve.
    \item Második körben szintén véletlenszerűen választunk egy pontot - az eddig meglevő 2 közül -, majd egy harmadik pontot kötünk egy éllel hozzá.
    \item A fentieket tetszőleges ideig folytatjuk. A szimulációt esetemben adott élszám elértéig futtattam.
\end{enumerate}
A második feladatban (\apm) megtett lépések a (\ref{sec:2})-es leírásban szereplőkkel megegyezőek.
\\ \\
A szimuláció futtatása mindkét feladat esetében összesen 3 darab adatsorral tért vissza. Ezek egy kezdeti értékkel voltak inicializálva, majd ezek minden lépésben frissültek, végül pedig a végleges értéküket használtam fel a további számításokban és ábrázolásokban. Ezek az adatsorok a következőek voltak, a szimuláció forráskodjában is használt nevük szerint:

\begin{itemize}
    \item[--] \texttt{graph\_rrt}: Amikor egy újabb pont érkezik a gráfba, az minden alkalommal pontosan 1 ponthoz csatolódik hozzá. Ez az \texttt{array} típusú adatsor sorrendben tartalmazza, hogy az újabb csúcsok melyik másik csúcshoz kapcsolódtak hozzá beérkezésükkor. Hossza egyenlő az összes él darabszámával, mely jelen esetben egyel kevesebb a rendszerben szereplő pontok végleges számánál.
    \item[--] \texttt{count\_rrt}: Ez a - szintén - \texttt{array} típusú adatsor tartalmazza, hogy az egyes pontok fokszámát. Az első eleme az első pont fokszáma, a második eleme a második pont fokszáma és így tovább. Hossza megegyezik az összes, rendszerben levő pont számával, ami pontosan egyel több, mint a rendszerben levő élek száma ebben az esetben.
    \item[--] \texttt{dist\_rrt}: Ez az \texttt{array} tartalmazza sorrendben az egyes fokszámmal rendelkező csúcsok számát. Első eleme a $0$ fokszámú csúcsok száma, második eleme az $1$ fokszámú csúcsok száma, és így tovább. Hossza egyenlő a maximális lehetséges fokszámmal. Ez megegyezik az összes él számával, ugyanis ebben a kis valószínűségű esetben minden csúcs ugyanahhoz az egyetlen ponthoz csatlakozik. Ekkor a központi csúcshoz csatlakozik az összes $N$ darab él, és így fokszáma is $N$.
\end{itemize}
A második modell esetén az eredetileg visszatérő adatsorok típusai az elsővel megegyezőek voltak, csak ott rendre \texttt{graph\_apm}, \texttt{count\_apm} és \texttt{dist\_apm} jelölésekkel. \\
Az első modell esetén az \texttt{rrt} a \emph{Random Recursive Tree}, míg a második esetben az \texttt{apm} rövidítés az \emph{Anti-Preferential Model} kifejezés rövidítése.

\subsection{Kiértékelés}
Mindkét modell egy-egy növekedő hálózat, melynek egyes tulajdonságait (pl. $P_{k}$ eloszlásfüggvény) már az órán kiszámoltuk. A továbbiakban azokra fogok alapozni. \\
A (\ref{fig:1}) és (\ref{fig:2}) ábrákon sorrendben a \rrt és az \apm alapvető karakterisztikáit ábrázoltam, $E=100$ él esetére. Nővekedő hálózat gyanánt ilyenkor $N = E+1 = 101$ csúcs lesz a hálózatban, hisz minden alkalommal egy beérkező fa, pontosan $1$ másik, már bent levő csúcshoz csatlakozik hozzá. \\
Mindkét ábrán a felső grafikon az egyes pontok végleges $k$ fokszámait mutatja, a hálózatba történő beérkezési sorrendjükben, $1$-től indexelve. A középső grafikonok az adott $k$ fokszámú pontok $N_{k}$ darabszámát mutatják, kezdve a $k=0$ fokszámtól egészen az $E=100$ esetében maximálisan lehetséges $k=100$-ig. Az alsó grafikonokon a $P_k = \frac{N_{k}}{N}$ diszkrét eloszlásfüggvényt ábrázoltam, mely tulajdonképpen csak a középső grafikonok normáltja.
\\ \\
Mivel az első feladatban a \rrt esetében vizsgálnunk kell ennek a diszkrét, mért $P_{k}$ függvénynek az elméletileg várt folytonos $P_{k}$ függvényhez képesti hibáját, így itt ezt a függvényt is ábrázoltam az alsó grafikonon, egy szaggatott szürke vonallal. A két eloszlásfüggvény külsőre nagyon hasonló, azonban két fő különbség észrevehető rajta. Az első, hogy a \rrt esetén a $k=1$ fokszámú csúcsok magasan a legnagyobb arányban szerepelnek a hálózatban, míg az \apm esetében minden alkalommal közel azonos a $k=1$ és $k=2$ csúcsok darabszáma. Ezek közül is legtöbb esetben a $k=2$ fokszámú csúcsok vezetnek számosság terén. A második különbség, hogy az \apm esetén a fokszámeloszlás sokkal \q{simább}, nincsenek a $0$ darabszámú esetek sorozatát elérve apró fluktuációk, ahogy az a \rrt esetében a (\ref{fig:1})-es ábrán is láthatóan történik.
\\ \\
Az órán kiszámoltuk, hogy a nővekedő hálózat elméleti fokszámeloszlása a következő:

\begin{equation}
    P_{k} = e^{-k \ln \left( 2 \right)}
\end{equation}
A szimuláció hibáját az elméletitől való százalékos eltérésnek vettem. Feladatunk, hogy meghatározzuk azt az $N$-t, ami felett az első 10 fokszám hibája mindig kisebb, mint $10\%$. Ezt meghatározandó, vizsgáltam az első 10 elem hibáinak maximumát az $N$ növelésének függvényében. Egy tetszőleges futásra ($N=40\,000$) esetén kapott adatsorokat a (\ref{fig:3})-as ábrán közlöm. A kapott eredmények analízise a képaláiratban olvashatóak részletesebben. Összefoglalóként, az ilyen módon vett hiba csak fluktuációk során csökken $10\%$ alá, máskülönben - feltételezhetően - ahhoz propagál végtelenül hosszú futásidő során.
\\ \\
A fokszámok átlagos értékének számára már jóval konkrétabb, és az elméletből is várt eredményt kaptam. Mind a \rrt, mind pedig az \apm növekedő hálózat mivoltából kiindulva tudjuk, hogy a $k$ fokszám várható értéke $\left< k \right> = 2$, melyet az alábbi számításból kaphatunk meg:

\begin{equation}
    \left< k \right>
    =
    \sum_{i = 1}^{\infty} k_{i} * P_{i}
    \equiv
    \sum_{k = 1}^{\infty} k * P_{k}
    =
    \sum_{k = 1}^{\infty} k * e^{-k \ln \left( 2 \right)}
    =
    \sum_{k = 1}^{\infty} k * \left( e^{\ln \left( 2 \right)} \right)^{-k}
    =
    \sum_{k = 1}^{\infty} k * 2^{-k}
\end{equation}
Mely egy számelméletből régóta ismert összeg. Írjuk át az alábbi formába a kapottakat:

\begin{equation}
    \sum_{k = 1}^{\infty} k * 2^{-k}
    =
    \sum_{k = 1}^{\infty} k * \left( \frac{1}{2} \right)^{k}
\end{equation}
Melyre alkalmazhatjuk az alábbi ismert összefüggést:
\begin{equation}
    \left| x \right| < 1:
    \quad
    f \left( x \right)
    =
    \sum_{n = 1}^{\infty} x^{n}
    =
    \frac{x}{1 - x}
\end{equation}
\begin{equation}
    xf' \left( x \right)
    =
    \sum_{n = 1}^{\infty} n * x^{n}
    =
    \frac{x}{(1 - x)^2}
\end{equation}
Ha most $n = k$ és $x = \frac{1}{2}$ értékeket veszünk, akkor felírhatjuk a következőt:

\begin{equation}
    \left< k \right>
    =
    \sum_{k = 1}^{\infty} k * \left( \frac{1}{2} \right)^{k}
    =
    \frac{\frac{1}{2}}{(1 - \frac{1}{2})^2}
    =
    \frac{\frac{1}{2}}{\frac{1}{4}}
    =
    2
\end{equation}
Tehát végtelenül sok él esetén azt várjuk, hogy az átlagos $k$ érték $2$-höz tart. Pontosan ezt figyelhetjük meg a szimulációban is, melyet a (\ref{fig:5})-ös és (\ref{fig:6})-os ábrán illusztráltam, rendre a \rrt és az \apm esetében.
\\ \\
Végül a maximális fokszámok alakulását kellett megvizsgálnunk, az $N$ csúcsok számának növelésének függvényében. Ezt a két modell esetében a (\ref{fig:7})-es és (\ref{fig:8})-as képeken ábrázoltam. A \rrt esetén megfigyelhető, hogy a maximális értékek a csúcsok/élszámok függvényében egy exponenciális, vagy esetleg gyökös növekedést mutatnak. Az \apm esetén a hosszú futásidő miatt csak $E=2000$ élig tudtam a szimulációt futtatni, itt még nem rajzolódik ki egyértelműen a \rrt esetében megfigyelhető mintázat.\\
A maximális fokszám $\left< k_{max} \right>$ átlagát több, azonos $N$ csúcsszámú futtatásra számítva határoztam meg. Ezek eredményét a (\ref{fig:9})-es ábrán közöltem. A maximális fokszámok élszámfüggéséhez hasonló módon itt is egyértelműen egy exponenciális, vagy gyökös növekedés rajzolódik ki. Az érték hibáját itt kiszámolni felesleges, ugyanis a növekedési tendencia bőven az értékek oszcillációjánál nagyobb nagyságrendileg.