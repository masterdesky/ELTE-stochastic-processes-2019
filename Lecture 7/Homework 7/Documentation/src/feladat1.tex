\section{} \label{sec:1}
K:\ Végezzünk szimulációkat az egy-dimenziós Ising modellre. A rendszer állapotát egydimenziós rács pontjaiban $\left( i = 1, 2,\ \dots, N \right)$ ülő $s_{i} = \pm 1$ spinek határozzák meg. A spinek szomszédjaikkal ferromágnesesen hatnak kölcsön, azaz egy állapot energiája a következő

\begin{equation*}
    E \left( s_{1}, s_{2},\ \dots,\ s_{N} \right)
    =
    - J \sum_{1}^{N-1} s_{i} s_{i+1}
    \quad , \quad
    J\ >\ 0
\end{equation*}
A spinek $T$ hőmérsékletű környezettel vannak kölcsönhatásban, s ennek eredményeképpen átbillenhetnek egyik állapotukból a másikba $\left( s_{i} \to -s_{i} \right)$. \\
Válasszunk spin-flip rátának olyan alakot, ami kielégíti a részletes egyensúly elvét. Ilyen lesz például, ha az $i$-edik spin forgatásának $\left( s_{i} \to -s_{i} \right)$ rátája a következő ($1/\text{s}$ egységben):

\begin{equation*}
    w_{i} \left( s_{1},\ \dots,\ s_{i-1}, s_{i}, s_{i+1},\ \dots,\ s_{N} \right)
    =
    \begin{cases}
        1   &\text{ha } \Delta E < 0 \\
        1/2 &\text{ha } \Delta E = 0 \\
        e^{- \beta T} &\text{ha } \Delta E > 0
    \end{cases}
\end{equation*}
ahol, mint könnyen belátható

\begin{equation*}
    \Delta E
    =
    2 J s_{i} \left( s_{i-1} + s_{i+1} \right)
\end{equation*}
Legyen $N = 100$, s kezdjük a rendszer szimulálását teljesen véletlenszerű állapotból, ahol
$1/2$ valószínűséggel $s_{i} = \pm 1$ (az egyensúlyi átlagok nem függhetnek a kezdeti feltételtől, tehát
eredményeink helyességét ellenőrizhetjük azzal, hogy teljesen rendezett állapotból indítjuk a
rendszert, s megnézzük ugyanazt kapjuk-e).
\\ \\
A szimulálás a következő lépésekből áll:

\begin{enumerate}
    \item \label{enum_obj:1}Véletlenszerűen kiválasztunk egy spint.
    \item \label{enum_obj:2}Megnézzük, hogy ha megforgatjuk, akkor mennyit változik a rendszer energiája, azaz kiszámítjuk $\Delta E$-t.
    \item \label{enum_obj:3}Ha $\Delta E < 0$, akkor megforgatjuk a spint, s megyünk az (\ref{enum_obj:1})-es ponthoz.
    \item \label{enum_obj:4}Ha $\Delta E = 0$, akkor húzunk egy véletlen számot $P$-t a $\left[ 0, 1 \right]$ intervallumból, s ha $P < 1/2$, akkor megforgatjuk a spint, s megyünk az (\ref{enum_obj:1})-es ponthoz. Ha $P > 1/2$, akkor forgatás nélkül megyünk az (\ref{enum_obj:1})-es ponthoz.
    \item \label{enum_obj:5}Ha $\Delta E > 0$, akkor húzunk egy véletlen számot $P$-t a $\left[ 0, 1 \right]$ intervallumból, s ha $P < e^{-\beta \Delta E}$, akkor megforgatjuk a spint, egyébként megyünk az (\ref{enum_obj:1})-es ponthoz.
\end{enumerate}
Az (\ref{enum_obj:1})-(\ref{enum_obj:5}) pontokat sokszor, $N * t$-szer elvégezve azt mondjuk, hogy $t$ idő telt el. Minden rendszernek van általában egy relaxációs ideje, $\tau$, s ha $t > \tau$, akkor a rendszer elérkezik az egyensúlyba, s attól kezdve a különböző mennyiségek, mint például a mágnesezettség

\begin{equation*}
    m
    =
    \frac{1}{N} \sum_{i=1}^{N} s_{i}
\end{equation*}
vagy a mágnesezettség fluktuációja $m^{2}$, az egyensúlyi értéke körül fluktuál. Az egyensúlyi átlagokat ($\left< m \right>$, $q\left< m^{2} \right>$) tehát megbecsülhetjük mint időátlagokat. Ez azt jelenti, hogy t1 időnként kiszámítjuk (megmérjük) az m és az m2 értékét, majd elég sok ilyen mérésből átlagokat számolunk, s ezek megadják a $T$ hőmérsékleti termodinamikai átlagokat, $\left< m \right>$-t és $\left< m^{2} \right>$-t. \\
Határozzuk meg az $\left< m \right>$ és az $\left< m^{2} \right>$ átlagokat az alábbiakban megadott egyéni $\beta J$ értékeknek megfelelő hőmérsékleteken. Próbáljuk megmagyarázni az eredményt! Mentsünk el egyegy egyensúlyi spinkonfigurációt $\left\{ s_{1},\ \dots, s_{i-1}, s_{i} , s_{i+1},\ \dots, s_{100} \right\}$ mind a négy $\beta J$ értéknél, egy későbbi feladatban szükség lesz rájuk.

\section*{\bfseries\large\MakeUppercase{Megoldás}}

\subsection{Elméleti alapok}
Az Ising-modell egy, a ferromágnességet statisztikus fizikai szempontból leíró matematikai modell. Vizsgálatának tárgyát olyan diszkrét elemek képezik, melyek atomi mágneses momentumokat/spineket szimbolizálnak. Ezen elemek $+1$, vagy $-1$ értéket vehetnek fel, melyek mindegyike a szomszédjával hathat kölcsön. Jelentősége, hogy ezen rendszer 2-dimenziós változata szolgáltatja az egyik legegyszerűbb statisztikus modellt a fázisátalakulások vizsgálatához. A feladat során egy ilyen rendszer időfejlődését kellett vizsgálnunk.
\\ \\
Vizsgáljuk $N$ darab atom viselkedését egy $z$-irányú $H$ mágneses térben. Ha a fenti definíció alapján a spinek $+1$ és $-1$ értéket vehetnek fel, akkor rendszer aktuális állapotának Hamilton-függvényéből kapott teljes energiája a következő formában írható:

\begin{equation} \label{eq:1}
    E \left( s_{1}, s_{2},\ \dots,\ s_{N} \right)
    =
    - J \sum_{\left< i,j \right>} s_{i} s_{j}
    -
    \mu H \sum_{i = 1}^{N} s_{i}
\end{equation}
Az eredeti leírás alapján a jelenlegi feladat egyik közelítése, hogy az általunk vizsgált rendszerben most $H = 0$, tehát a második tag zérus. Így a teljes energia a leírásban is látható módon fejezhető ki:

\begin{equation} \label{eq:2}
    E \left( s_{1}, s_{2},\ \dots,\ s_{N} \right)
    =
    - J \sum_{\left< i,j \right>} s_{i} s_{j}
\end{equation}
Ahol $J$ az ún. \emph{kicserélődési energia}, ${\left< i,j \right>}$ pedig a szomszédos elemeken végigfutó összegzést jelenti. Az 1-dimenziós esetben ez a fenti $\sum_{i=1}^{N-1} s_{i} s_{i+1}$ alakot veszi fel. Megjegyzendő, hogy ez az alak azt takarja, hogy minden párt csupán egyetlen egyszer veszünk bele a teljes energiába, triviális módon. Két dimenziós, zárt határfeltételű rendszer esetében pedig az energiát a $\sum_{i=1}^{N-1}\sum_{j=1}^{N-1} s_{i,j} \left( s_{i+1,j} + s_{i,j+1} + s_{i+1,j+1} \right)$ alakba írhatjuk át, biztosítva az 1-dimenziós esetben is használt feltételt, miszerint minden párt csak egyszer számolunk.
\\ \\
Feladataink közé tartozott monitorozni a mágnesezettség $m$ fejlődését, valamint vizsgálni annak és négyzetének várható értékét, mint időátlagot. Valójában az $\left< m \right>$ és $\left< m^{2} \right>$ átlagolás sokaságátlag, több különböző véletlen futtatás eredménye, azonban ezt egyensúlyban jól közelíti az időátlag. Célunk tehát egy olyan rendszert létrehozni, melyet tetszőleges pontból indítva az eléri a - kezdőfeltételektől  független - egyensúlyi állapotát, majd akörül fluktuál és így vizsgálhatóvá válnak az egyes mennyiségek várható értékei.
\\ \\
Az $\left< m \right>$ és $\left< m^{2} \right>$ elméletileg várt értékét könnyen kiszámíthatjuk. Írjuk fel ezen két mennyiséget a várható érték és a $P$ állapot-valószínűség definíciója alapján:

\begin{equation} \label{eq:3}
    \left< m \right>
    =
    \sum_{\sigma} m \left( s_{1},\ \dots, s_{i-1}, s_{i} , s_{i+1},\ \dots, s_{N} \right)
    *
    P_{\beta} \left( s_{1},\ \dots, s_{i-1}, s_{i} , s_{i+1},\ \dots, s_{N} \right)
\end{equation}
\begin{equation} \label{eq:4}
    \left< m^{2} \right>
    =
    \sum_{\sigma} m^{2} \left( s_{1},\ \dots, s_{i-1}, s_{i} , s_{i+1},\ \dots, s_{N} \right)
    *
    P_{\beta} \left( s_{1},\ \dots, s_{i-1}, s_{i} , s_{i+1},\ \dots, s_{N} \right)
\end{equation}
Ahol a szummázás az összes spinkonfiguráción fut végig, $\sigma$-val jelölve azokat. Itt a $P_{\beta}$ valószínűséget a következő módon definiáljuk:

\begin{equation} \label{eq:5}
    P_{\beta} \left( s_{1},\ \dots, s_{i-1}, s_{i} , s_{i+1},\ \dots, s_{N} \right)
    =
    \frac{1}{Z_{\beta}} e^{- \beta E \left( s_{1},\ \dots, s_{i-1}, s_{i} , s_{i+1},\ \dots, s_{N} \right)}
\end{equation}
Ahol $Z_{\beta}$ a $\beta$ inverzhőmérsékletű rendszer állapotösszege. Jelöljük $\left( s_{1},\ \dots, s_{i-1}, s_{i} , s_{i+1},\ \dots, s_{N} \right)$ tagot az egyszerűség kedvéért a fentebb is már használt $\left( \sigma \right)$-val. Ekkor felírhatjuk a következőt:

\begin{equation} \label{eq:6}
    \left< m \right>
    =
    \sum_{\sigma} m \left( \sigma \right) * \frac{1}{Z_{\beta}} e^{- \beta E \left( \sigma \right)}
    =
    \sum_{\sigma} m \left( \sigma \right) * \frac{1}{Z_{\beta}} e^{+ \beta J \sum_{\left< i,j \right>} s_{i} s_{j}}
\end{equation}
\begin{equation} \label{eq:7}
    \left< m^{2} \right>
    =
    \sum_{\sigma} m^{2} \left( \sigma \right) * \frac{1}{Z_{\beta}} e^{- \beta E \left( \sigma \right)}
    =
    \sum_{\sigma} m^{2} \left( \sigma \right) * \frac{1}{Z_{\beta}} e^{+ \beta J \sum_{\left< i,j \right>} s_{i} s_{j}}
\end{equation}
Várakozásaink szerint az energia egyensúlyi állapotban $0$ körül fluktuál, így $\beta J \sum_{\left< i,j \right>} s_{i} s_{j}$ értéke szintén $0$ várható értékkel rendelkezik, és így $\left< m \right>$ is tartani fog $0$-hoz. A mágnesezettségek négyzetének várható értéke, $\left< m^{2} \right>$ viszont kvázi csak $\beta$ értékétől $Z_{\beta}$-n keresztül, valamint $m^{2}$ menetétől fognak függeni. Mivel $m^{2}$ gyorsan eléri a nullát és azon érték felett \q{pattog}, ezért $\left< m^{2} \right>$ egy valamivel $0$ feletti érték lesz hosszú idő után. Ez azt jelenti, hogy ha ezen átlag időfejlődését vizsgáljuk, akkor az egy konstans görbéhez fog tartani.
\\ \\
A kiszámolt átlagértékek pontosságát az átlagértékek standard hibájának számításával definiáltam, mely a következő mennyiség:

\begin{equation} \label{eq:8}
    \sigma_{\left< x \right>}
    =
    \frac{\sigma_{x}}{\sqrt{n}}
\end{equation}
Ahol $\sigma_{x}$ az $x$ adat szórása, $n$ pedig az adatpontok száma.

\subsection{Megvalósítás}
A szimulációt az 1D esetre kellett megírnunk, majd abban vizsgálni a fentebb tárgyalt mennyiségek időfejlődését is. Az ellenőrzés és az esetleges mélyebb megértés kedvéért én a szimulációban a 2D esetet is implementáltam. A programkódot egy Jupyter Notebook-ban futó Python 3.7 kernel alatt írtam, az ezen jegyzőkönyvhöz készült ábrákat pedig szintén abban a notebook-ban készítettem. Minden felhasznált kód, és maga a dokumentáció is elérhető GitHub-on\cite{github}. A 2D esetben készült időfejlődésről animációt is készítettem, melyet elérhetővé tettem a YouTube-on\cite{yt}. \\
Minden szimulációt homogén kezdeti feltétellel, minden spin értékét egyaránt $+1$-nek választva indítottam és úgy vizsgáltam annak időbeli alakulását. Ennek oka az volt, hogy így le tudtam ellenőrizni, hogy egy szélső feltételből is az egyensúlyi helyzetbe propagál idővel a rendszer és ott is marad-e.

\subsection{1D Ising-modell}
Az 1D Ising-modellről készült grafikonokat az (\ref{fig:1}) - (\ref{fig:10}) ábrákon közöltem. Az (\ref{fig:1})-es ábra az egyes $\beta$ értékekkel futtatott végállapotokat mutatja, $N = 200\,000$ szimulációs lépés után, $300$ db spinre. Míg a (\ref{fig:2}) ábrán ugyanezen szimulációk során mért energia időbeli változását ábrázoltam. \\
A kapott eredmények egyértelműen mutatják több futtatás után is, hogy az egyensúlyi helyzet akkor áll be, amikor a spinek tökéletesen össze vannak keveredve, tehát egyenlő számban tartalmazva $+1$ és $-1$ spineket. A rendszer ezt az állapotot elérve ekörül oszcillál. A mágnesezettség ebben az állapotban szintén $0$ körül fluktuál, ami a (\ref{fig:3})-as ábráról olvasható le. A (\ref{fig:4})-es ábrán ez a fluktuáció látható kiemelve. \\
Összefoglalva úgy fogalmazhatjuk meg, hogy tér nélkül egy anyag nem szeret mágnesezett állapotban maradni. Ha a spinek egymással kölcsönhatva, de szabadon elfordulhatnak, a mágnesezettség minden esetben $0$-hoz fog tartani, így biztosítva az egyensúlyt. \\
Ennek megfelelően - szintén a várakozásainkkal megegyező módon - az $m^{2}$ érték valamivel a $0$ felett fluktuál, melyek a (\ref{fig:5}) és (\ref{fig:6}) ábrákon láthatóak. A (\ref{fig:6})-os ábrán a (\ref{fig:5})-ösön is látható nulla feletti fluktuáció van kiemelve.
\\ \\
A (\ref{fig:7}) és (\ref{fig:8}) ábrákon sorrendben a mágnesezettség és a mágnesezettség négyzet időátlagának változását ábrázoltam a lépéshossz függvényében, melyet ezen mennyiségek várható értékének megbecsüléséhez használtam. A feladat leírásában is szerepel megjegyzésként, miszerint a mágnesezettség értéke csak az egyensúlyi helyzetben értelmezett. Ezt figyelembe véve, egy manuálisan választott pontot kijelöltem az egyensúlyi helyzet alsó határának, és kizárólag az időben utána következő értékeket használtam fel a mágnesezettség és annak négyzetének időátlagát megadó számításokban. Tökéletesen pontos értéket akkor kapnánk mindkét mennyiségre, ha a szimulációt végtelenül hosszú ideig futtatnánk. Ekkor az ábrázolt $\left< m \right> \left( n \right)$ és $\left< m^{2} \right> \left( n \right)$ lépések számától függő görbe legutolsó pontja lenne a pontos becslésünk. Mivel ezt megvalósítani nem lehetséges, így szimplán viszonylag hosszú ideig futtattam a szimulációt, majd meghatároztam a (\ref{eq:8}) képletben is szereplő módszerrel az utolsó pont értékének hibáját. Ez az általam adott becslés az $\left< m \right>$ és $\left< m^{2} \right>$ értékekre. Ezekre a következő eredményeket kaptam:

\begin{center}
\begin{tabular}{c||c|c}
    \hline
    Hőmérsékletek [K] & $\left< m \right>$  & $\left< m^{2} \right>$ \\ \hline\hline
    301.79            & $0.000 \pm 0.0001$  & $0.0033 \pm 0.00001$   \\ \hline
    150.90            & $0.002 \pm 0.0001$  & $0.0031 \pm 0.00001$   \\ \hline
    48.29             & $0.004 \pm 0.0001$  & $0.0033 \pm 0.00001$   \\ \hline
    22.63             & $-0.004 \pm 0.0001$ & $0.0032 \pm 0.00001$   \\ \hline
\end{tabular}
\end{center}
\captionof{table}{\normalfont Az $\left< m \right>$ és $\left< m^{2} \right>$ átlagok értékei}\label{tab:1}
\hfill \break
Az ezeket bemutató grafikonok sorrendben a (\ref{fig:7}) és (\ref{fig:8}) ábrákon láthatóak. \\
Alacsony hőmérsékleten az anyag hajlamos ún. \q{beakadni} egy szélsőségesebb spinállapotba. Ennek oka, hogy a szomszédos spinek szeretnek azonos állapotban tartózkodni, azonban alacsony hőmérsékleten az átfordulás rátája is nagyon kicsi. Ez azt eredményezi, hogy nagyobb arányban jönnek létre azonos spinállapotban levő domének az anyagok belül, így az $\left< m \right>$ átlag eltolódik véges távolságon vagy a pozitív, vagy a negatív irányba, ami mind a fenti táblázat adatain, mind pedig a grafikonokon jól látszik. Természetesen végtelenül hosszú időre az átlag ebben az esetben is a $0$-hoz tart $\left< m \right>$ esetén. Ezzel ellentétben magasabb hőmérsékleteken ilyen viselkedést már nem egyre kevésbé lehet megfigyelni, a mágnesezettség átlaga közel $0$ körül mozog végig már rövid idő után is.
\\ \\
Továbbiakban vizsgáltam az $\left< m^{2} \right> - \left< m \right>^{2}$ értékét is. Ennek eredményét a (\ref{fig:9})-es ábrán közöltem. A kapott értékeket az alábbi táblázatban foglaltam össze:
\begin{center}
\begin{tabular}{c||c}
    \hline
    Hőmérsékletek [K] & $\left< m^{2} \right>$ - $\left< m \right>^{2}$ \\ \hline\hline
    301.79            & $0.0036 \pm 0.00006$                            \\ \hline
    150.90            & $0.0036 \pm 0.00006$                            \\ \hline
    48.29             & $0.0034 \pm 0.00006$                            \\ \hline
    22.63             & $0.0034 \pm 0.00006$                            \\ \hline
\end{tabular}
\end{center}
\captionof{table}{\normalfont Az $\left< m^{2} \right> - \left< m \right>^{2}$ fluktuációk értékei}\label{tab:1}
\hfill \break

\subsection{2D Ising-modell}
A feladat elvégzése után a 2D Ising-modellt is megvizsgáltam, csupán érdekesség gyanánt. Ebben egy 2D rács pontjaiba szórtam le diszkrét, kezdetben teljesen homogén módon $+1$ spinértékeket, és azok időfejlődését vizsgáltam a fentiekhez hasonló módon $100\,000$ lépésre, mindenféle mágneses tér jelenléte nélkül. Az ebben az esetben készült grafikonok megtekinthetőek a (\ref{fig:12}) - (\ref{fig:22}) ábrákon, az 1D Ising-modellel megegyező sorrendben és azonos vizsgálati módszerek szerint. Az egyensúlyi helyzet itt is akkor állt be, amikor a spinállapot nagyjából egyenlő arányban $+1$ és $-1$ spinekből tevődött össze, és ekkor a mágnesezettség is $0$ körül oszcillált.
\\ \\
Az erről készült animációt\cite{yt} homogén $+1$ értékű spinekkel teli állapotból indítottam, és $10000$ lépés hosszan futtattam. Végeredményben vizuálisan is a fent tárgyalt eredmények látszódnak a videón.