\section{} \label{sec:2}
\subsection*{A) feladatrész}
\quest{Dobjuk fel az érmét kétszer. Milyen valószínűséggel kapunk két fejet (FF), illetve írás-fej (IF) sorrendet? Ugyanaz a két valószínűség?}
\\ \\
Számoljunk a $\frac{\text{kedvező}}{\text{összes}}$ szabállyal jelen esetben. Az első alkalommal azt mondhatjuk, hogy a FF dobás valószínűsége a követező:

\begin{equation} \label{eq:1}
    \frac{\text{\{FF\}}}{\{FF; IF; FI; II\}} = \frac{1}{4}
\end{equation}
Ugyanis összesen négy különböző eset lehetséges, ezekből mi az egyiket várjuk eredményül. Második esetben ugyanezt mondhatjuk el, hasonlóan írhatjuk fel az IF dobás valószínűségét:

\begin{equation} \label{eq:2}
    \frac{\text{\{IF\}}}{\{FF; IF; FI; II\}} = \frac{1}{4}
\end{equation}
Szintén egy lehetőséget választunk ki az összesen várható négy közül.
\\ \\
Másképp is leírhatjuk a helyzetet. Ismert, hogy mind a fej, mind az írás dobásának valószínűsége

\begin{equation} \label{eq:3}
    P \left( \text{fej (F)} \right) = P \left( \text{írás (I)} \right) = \frac{1}{2}
\end{equation}
Mivel a pénz második feldobása az első dobástól független esemény, ezért felírhatjuk, hogy:

\begin{equation} \label{eq:4}
    P \left( \text{FF} \right) = \frac{1}{2} * \frac{1}{2} = \frac{1}{4}
\end{equation}
Ugyanígy a másik esetre:

\begin{equation} \label{eq:5}
    P \left( \text{IF} \right) = \frac{1}{2} * \frac{1}{2} = \frac{1}{4}
\end{equation}
A két esemény tehát azonos valószínűséggel fordul elő.

\subsection*{B) feladatrész}
\quest{Játsszuk a következő játékot! Addig dobálunk, amíg vagy két fej (FF - én nyerek), vagy fej-írás (FI - te nyersz) jön ki. Igazságos ez a játék?}
\\ \\
A nyerés feltétele mindkét játékos számára, hogy az első dobás fej (F) legyen. Ezt követően akkor nyer vagy az egyik vagy a másik, ha vagy fej (F), vagy írás (I) a rá következő dobás. Itt is elmondhatjuk hogy az egymást követő dobások független események, így egy F eredmény követően mind egy F, mind pedig egy I azonosan $\frac{1}{2}$ valószínűséggel következik be. \\
Azt mondhatjuk tehát, hogy igen, igazságos a játék, ugyanis mindkét fél nyerési esélye azonos. Azt feltételezni az ilyen események sorozatánál, hogy a második dobás függ az előtte levőtől (pl. hogy egy F után $\frac{1}{4}$ valószínűséggel következik be még egy F és így az írás előnyben van) szokás a \emph{szerencsejátékosok tévedésének}\cite{croson2005gambler}, vagy \emph{Monte Carlo tévedésnek} hívni. Ez arra a rossz meglátásra alapul, miszerint egy már (gyakran) előfordult eseményről intuitíve azt gondoljuk, hogy a jövőben kisebb valószínűséggel fog előfordulni. Ez független események esetén viszont nem igaz, lásd a fenti pénzdobás példája. Mindegy hányszor fordult elő már F, vagy I a dobások során, a következő esetében mind F, mind pedig I azonosan $P = \frac{1}{2}$ valószínűséggel következik be.