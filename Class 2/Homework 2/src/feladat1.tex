\section{} \label{sec:1}
\quest{A Perrin kísérlet megértéséhez először oldjuk meg a két-dimenziós Brown mozgás következő változatát:\\
Egy $l$ rácsállandójú négyzetrácson egy részecske $\tau$ időközönként, egyenlő valószínűséggel ugrik a négy szomszédos rácspont egyikébe, s az egymást követő lépések függetlenek egymástól. A részecske az $(x_{0} = 0, y_{0} = 0)$ pontból indul. \\
Határozzuk meg a $t = N \tau$ idő alatti várható elmozdulást, $\sqrt{\left< r^{2} \right>} = \sqrt{\left< x_{t}^{2} \right> + \left< y_{t}^{2} \right>}$-t!}

\subsection{Matematikai megfontolások}
Keressük az $\left< x_{t}^{2} \right>$ és az $\left< y_{t}^{2} \right>$ értékeket, amikből aztán megkaphatjuk a keresett $\sqrt{\left< r^{2} \right>}$ mennyiséget. Ezeknek definíciója ismert:

\begin{equation} \label{eq:1}
    \left< x_{t}^{2} \right>
    =
    \int_{\ -\infty}^{\ \infty} x^{2} * P \left( x \right)\ dx
    \equiv
    \int_{\ -\infty}^{\ \infty} x^{2} * P \left( x, t \right)\ dx
\end{equation}
\begin{equation} \label{eq:2}
    \left< y_{t}^{2} \right>
    =
    \int_{\ -\infty}^{\ \infty} y^{2} * P \left( y \right)\ dy
    \equiv
    \int_{\ -\infty}^{\ \infty} y^{2} * P \left( y, t \right)\ dy
\end{equation}
Jelen esetben a 2D Brown--mozgás leírását kell megadnunk, amely esetében egy részecske, az adott pontban történő tartózkodásának valószínűségét egy $P \left( x, y \right) \equiv P \left( x, y, t \right)$ mennyiség jellemzi. Ekkor az $\left< x^{2} * y^{2} \right>$ megadhatjuk a következő módon:

\begin{equation} \label{eq:3}
    \left< x^{2} * y^{2} \right>
    =
    \iint{x^{2} * y^{2} * P \left( x, y \right)\, dx\, dy}
\end{equation}
Mivel $x$ és $y$ független események, ezért $P \left( x, y \right)$ felbontható egy $f_{1} \left( x \right)$ és egy $f_{2} \left( y \right)$ függvény szorzatára a következőek alapján:

\begin{equation}
    P \left( x, y \right) = f_{1} \left( x \right) * f_{2} \left( y \right) \quad \text{ha $x$ és $y$ független}
\end{equation}
Ekkor a fenti (\ref{eq:3})-as egyenlet így alakul:

\begin{equation} \label{eq:4}
    \iint{x^{2} * y^{2} * P \left( x, y \right)\, dx\, dy}
    =
    \iint{x^{2} * y^{2} * f_{1} \left( x \right) * f_{2} \left( y \right)\, dx\, dy}
\end{equation}
Ezt azonos változók szerint ketté bonthatunk és bevezethetünk egy többi változótól független időfüggőséget is, mely végeredménye a következő:

\begin{align} \label{eq:5}
    &\left< x^{2} * y^{2} \right>
    =
    \iint{x^{2} * y^{2} * f_{1} \left( x \right) * f_{2} \left( y \right)\, dx\, dy} = \nonumber \\
    =
    \int x^{2} * f_{1} \left( x \right)\, &dx * \int{y^{2} * f_{2} \left( y \right)\, dy}
    \equiv
    \underbrace{\int{x^{2} * f_{1} \left( x, t \right)\, dx}}_{\text{\normalfont $\left< x^{2} \right>$}} * \underbrace{\int{y^{2} * f_{2} \left( y, t \right)\, dy}}_{\text{\normalfont $\left< y^{2} \right>$}}
\end{align}

\subsection{}