\section{}
\subsection*{A) feladatrész}
\quest{Vizsgáljuk a Brown mozgás előadáson tárgyalt, Einstein-féle leírását, s legyen sodródás is a rendszerben (szél fúj a víz felett). \\
Ekkor a $\tau$ időnként megtett ugrások hosszának ($\Delta$) valószínűségi eloszlása nem szimmetrikus $\Phi \left( - \Delta \right) \neq \Phi \left( \Delta \right)$, s várhatóan $\overline{\Delta} = \int{\Delta \Phi \left( \Delta \right)\, d \Delta} \neq 0$. \\
Írjuk fel a Chapman-Kolmogorov egyenletet, s deriváljuk a részecske megtalálási valószínűségét, $P(x, t)$-t meghatározó Fokker-Planck egyenletet! Mennyiben különbözik ez az egyenlet az előadáson tárgyalt diffúziós egyenlettől?}
\\ \\
A feladat megoldása a \ref{subsec:3.2}-es fejezetben ismertetettekkel nagyrészt analóg. Az Brown-mozgás Einstein-féle leírásának feltételei sodródás esetén a következőek - ahogy többek között azok feladat szövegében is szerepelnek:

\begin{enumerate}
    \item $\int_{-\infty}^{\infty} \Phi \left( \Delta \right) d \Delta = 1$
    \item $\Phi \left( - \Delta \right) \neq \Phi \left( \Delta \right)$
    \item $\overline{\Delta} = \left< \Delta \right> = \int_{- \infty}^{\infty} \Delta \Phi \left( \Delta \right)\, d \Delta \neq 0$
\end{enumerate}
Annak valószínűsége, hogy egy részecske $\tau$ idő múlva az $x$ és $x + dx$ közötti tartományban foglal helyet:

\begin{equation}
    P \left( x, t + \tau \right) dx
    =
    P \left( x, t \right) dx
    -
    \int_{-\infty}^{\infty} \Phi \left( \Delta \right) d \Delta * P \left( x, t \right) dx
    +
    \int_{-\infty}^{\infty} \Phi \left( \Delta \right) d \Delta * P \left( x - \Delta, t \right) dx
\end{equation}
\begin{itemize}
    \item[--] A jobb oldal első tagja ($P \left( x, t \right) dx$) annak a valószínűségét jelöli, hogy a részecske már $t$ időpillanatban is az $x + dx$ helyen tartózkodott.
    \item[--] A második tag ($- \int_{-\infty}^{\infty} \Phi \left( \Delta \right) d \Delta * P \left( x, t \right) dx$) ebből levonódik, ugyanis ez annak a valószínűségét adja meg, hogy a részecske $\tau$ idő alatt kidiffundál az $x + dx$ tartományból.
    \item[--] A harmadik tag ($\int_{-\infty}^{\infty} \Phi \left( \Delta \right) d \Delta * P \left( x - \Delta, t \right)$) annak a valószínűségét jelenti, hogy egy részecske valahonnan pont a $d + dx$ tartományba ugrik bele $\tau$ időn belül.
\end{itemize}

Az egyenletet rendezzük, figyelve arra, hogy az integrálások $d \Delta$ szerint történnek. Emiatt minden tag, ami nem függ $\Delta$-tól, kiemelhető az integráljelek elé. A következőt kapjuk:

\begin{equation}
    P \left( x, t + \tau \right) dx
    =
    P \left( x, t \right) dx
    -
    P \left( x, t \right) dx * \left( \int_{-\infty}^{\infty} \Phi \left( \Delta \right) d \Delta \right)
    +
    dx * \left( \int_{-\infty}^{\infty} \Phi \left( \Delta \right) d \Delta * P \left( x - \Delta, t \right) \right)
\end{equation}
A $dx$ tagokkal az egyenlet leosztható, így:

\begin{equation}
    P \left( x, t + \tau \right)
    =
    P \left( x, t \right)
    -
    P \left( x, t \right) * \left( \int_{-\infty}^{\infty} \Phi \left( \Delta \right) d \Delta \right)
    +
    \int_{-\infty}^{\infty} \Phi \left( \Delta \right) d \Delta * P \left( x - \Delta, t \right)
\end{equation}
Alkalmazzunk a fent ismert három feltétel közül az első számút, miszerint $\int_{-\infty}^{\infty} \Phi \left( \Delta \right) d \Delta = 1$:

\begin{equation}
    P \left( x, t + \tau \right)
    =
    P \left( x, t \right)
    -
    P \left( x, t \right) * \underbrace{\left( \int_{-\infty}^{\infty} \Phi \left( \Delta \right) d \Delta \right)}_{\text{\normalfont =1, az első feltétel szerint.}}
    +
    \int_{-\infty}^{\infty} \Phi \left( \Delta \right) d \Delta * P \left( x - \Delta, t \right)
\end{equation}
\begin{equation}
    P \left( x, t + \tau \right)
    =
    \underbrace{P \left( x, t \right)
    -
    P \left( x, t \right)}_{\text{\normalfont =0}}
    +
    \int_{-\infty}^{\infty} \Phi \left( \Delta \right) d \Delta * P \left( x - \Delta, t \right)
\end{equation}
Mely után végül megkapjuk a \emph{Chapman--Kolmogorov-egyenletet}:

\begin{equation}
    \boxed{P \left( x, t + \tau \right)
    =
    \int_{-\infty}^{\infty} \Phi \left( \Delta \right) d \Delta * P \left( x - \Delta, t \right)}
\end{equation}
Az egyenlet bal oldalát $\tau$-ban, a jobboldalt pedig kétszer $\Delta$-ban sorbafejtjük (\emph{Kramers--Moyal-sorfejtés}). Ekkor a következő formulát kapjuk:

\begin{align}
    P \left( x, t \right) + \tau \frac{\partial P \left( x, t \right)}{\partial t}
    &=
    P \left( x, t \right) * \int_{-\infty}^{\infty} \Phi \left( \Delta \right) d \Delta
    -
    \frac{\partial P \left( x, t \right)}{\partial x} * \int_{-\infty}^{\infty} \Delta \Phi \left( \Delta \right) d \Delta + \nonumber \\
    &+
    \frac{1}{2} \frac{\partial^{2} P \left( x, t \right)}{\partial x^{2}} * \int_{-\infty}^{\infty} \Delta^{2} \Phi \left( \Delta \right) d \Delta
\end{align}
Amit az első számú feltétel alapján újfent tovább tudunk egyszerűsíteni:

\begin{align}
    \cancel{P \left( x, t \right)} + \tau \frac{\partial P \left( x, t \right)}{\partial t}
    &=
    \cancel{P \left( x, t \right)} * \overbrace{\int_{-\infty}^{\infty} \Phi \left( \Delta \right) d \Delta}^{\text{\normalfont =1, az első feltétel szerint.}}
    -
    \frac{\partial P \left( x, t \right)}{\partial x} * \int_{-\infty}^{\infty} \Delta \Phi \left( \Delta \right) d \Delta + \nonumber \\
    &+
    \frac{1}{2} \frac{\partial^{2} P \left( x, t \right)}{\partial x^{2}} * \int_{-\infty}^{\infty} \Delta^{2} \Phi \left( \Delta \right) d \Delta
\end{align}
Így végül a következő alakot kapjuk:

\begin{equation}
    \tau \frac{\partial P \left( x, t \right)}{\partial t}
    =
    -
    \frac{\partial P \left( x, t \right)}{\partial x} * \int_{-\infty}^{\infty} \Delta \Phi \left( \Delta \right) d \Delta
    +
    \frac{1}{2} \frac{\partial^{2} P \left( x, t \right)}{\partial x^{2}} * \int_{-\infty}^{\infty} \Delta^{2} \Phi \left( \Delta \right) d \Delta
\end{equation}
Ebben megjelenik két ismert tag, amiknek definícióját a (\ref{eq:6})-os és (\ref{eq:7})-es egyenletek adják meg:

\begin{equation}
    \left< \Delta \right> = \int_{-\infty}^{\infty} \Delta \Phi \left( \Delta \right) d \Delta
\end{equation}
\begin{equation}
    \left< \Delta^{2} \right> = \int_{-\infty}^{\infty} \Delta^{2} \Phi \left( \Delta \right) d \Delta
\end{equation}
Ezeket behelyettesítve és $\tau$-val leosztva kapjuk a következő \emph{Fokker--Planck-egyenletet}:

\begin{equation}
    \boxed{\frac{\partial P \left( x, t \right)}{\partial t}
    =
    -
    \frac{\left< \Delta \right>}{\tau} \frac{\partial P \left( x, t \right)}{\partial x}
    +
    \frac{1}{2} \frac{\left< \Delta^{2} \right>}{\tau} \frac{\partial^{2} P \left( x, t \right)}{\partial x^{2}}}
\end{equation}
Ez az egyenlet abban különbözik az - előadáson is megoldott - driftmentes változattól, hogy itt a második számú feltétel szerint $\left< \Delta \right> \neq 0$.


\subsection*{B) feladatrész}
\quest{Írjuk fel az egyenlet megoldását arra az esetre, ha a virágporszem az origóból indul!}
\\ \\
Az előző feladatban már ismertettük, hogy mit jelent egy rendszer \emph{driftje} és \emph{diffúziós együtthatója}, vezessük be itt is ugyanezeket a mennyiségeket:

\begin{equation}
    \frac{\left< \Delta \right>}{\tau} := v
\end{equation}
\begin{equation}
    \frac{1}{2} \frac{\left< \Delta^{2} \right>}{\tau} := D
\end{equation}
Ez visszahelyettesítve a Fokker--Planck-egyenletbe:

\begin{equation}
    \frac{\partial P \left( x, t \right)}{\partial t}
    =
    -
    v \frac{\partial P \left( x, t \right)}{\partial x}
    +
    D \frac{\partial^{2} P \left( x, t \right)}{\partial x^{2}}
\end{equation}
Ezt a differenciálegyenletet a $P \left( x, t = 0 \right) = \delta \left( x \right)$ kezdőfeltétellel oldjuk meg, ugyanis $t = 0$-ban a részecske az origóban tartózkodik. \\
Vezessük be az előző feladatból már ismert változócserét:

\begin{equation*}
    P \left( x, t \right)
    :=
    \tilde{P} \left( y \left( x, t \right), t \right)
    =
    \tilde{P} \left( x - vt, t \right)
\end{equation*}
Ekkor a következőképp módosul a fenti Fokker--Planck-egyenlet:

\begin{equation}
    \frac{\partial \tilde{P} \left( y \left( x, t \right), t \right)}{\partial t}
    =
    -
    v \frac{\partial \tilde{P} \left( y \left( x, t \right), t \right)}{\partial x}
    +
    D \frac{\partial^{2} \tilde{P} \left( y \left( x, t \right), t \right)}{\partial x^{2}}
\end{equation}
Ez teljes mértékben megegyezik az előző feladatban megoldott, driftelő rendszert leíró egyenlettel. Az egyenlet megoldásához ugyanazokat a lépéseket kell elvégezzük:
\begin{align}
    &\frac{\partial y \left( x, t \right)}{\partial t} * \frac{\partial \tilde{P} \left( y \left( x, t \right), t \right)}{\partial y}
    +
    \frac{\partial t}{\partial t} * \frac{\partial \tilde{P} \left( y \left( x, t \right), t \right)}{\partial t}
    = \nonumber \\
    =&
    -
    v * \frac{\partial y \left( x, t \right)}{\partial x} * \frac{\partial \tilde{P} \left( y \left( x, t \right), t \right)}{\partial y}
    +
    D * \partial_{x} \left( \frac{\partial y \left( x, t \right)}{\partial x} * \frac{\partial \tilde{P} \left( y \left( x, t \right), t \right)}{\partial y} \right)
\end{align}
\hrulefill
\begin{align}
    &\frac{\partial y \left( x, t \right)}{\partial t} * \frac{\partial \tilde{P} \left( y \left( x, t \right), t \right)}{\partial y}
    +
    \frac{\partial t}{\partial t} * \frac{\partial \tilde{P} \left( y \left( x, t \right), t \right)}{\partial t}
    = \nonumber \\
    =&
    -
    v * \frac{\partial y \left( x, t \right)}{\partial x} * \frac{\partial \tilde{P} \left( y \left( x, t \right), t \right)}{\partial y}
    +
    D * \partial_{x} \left( \frac{\partial \left( x - vt \right)}{\partial x} * \frac{\partial \tilde{P} \left( y \left( x, t \right), t \right)}{\partial y} \right)
\end{align}
\hrulefill
\begin{align}
    &\frac{\partial \left( x - vt \right)}{\partial t} * \frac{\partial \tilde{P} \left( y \left( x, t \right), t \right)}{\partial y}
    +
    1 * \frac{\partial \tilde{P} \left( y \left( x, t \right), t \right)}{\partial t}
    = \nonumber \\
    =&
    -
    v * \frac{\partial \left( x - vt \right)}{\partial x} * \frac{\partial \tilde{P} \left( y \left( x, t \right), t \right)}{\partial y}
    +
    D * \frac{\partial \left( x - vt \right)}{\partial x} * \frac{\partial \left( x - vt \right)}{\partial x} * \frac{\partial^{2} \tilde{P} \left( y \left( x, t \right), t \right)}{\partial y^{2}}
\end{align}
\hrulefill
\begin{align}
    &\underbrace{\frac{\partial \left( x - vt \right)}{\partial t}}_{\text{\normalfont $= -v$}} * \frac{\partial \tilde{P} \left( y \left( x, t \right), t \right)}{\partial y}
    +
    1 * \frac{\partial \tilde{P} \left( y \left( x, t \right), t \right)}{\partial t}
    = \nonumber \\
    =&
    -
    v * \underbrace{\frac{\partial \left( x - vt \right)}{\partial x}}_{\text{\normalfont $= 1$}} * \frac{\partial \tilde{P} \left( y \left( x, t \right), t \right)}{\partial y}
    +
    D * \underbrace{\frac{\partial \left( x - vt \right)}{\partial x}}_{\text{\normalfont $= 1$}} * \underbrace{\frac{\partial \left( x - vt \right)}{\partial x}}_{\text{\normalfont $= 1$}} * \frac{\partial^{2} \tilde{P} \left( y \left( x, t \right), t \right)}{\partial y^{2}}
\end{align}
\hrulefill
\begin{equation}
    \cancel{-v * \frac{\partial \tilde{P} \left( y \left( x, t \right), t \right)}{\partial y}}
    +
    \frac{\partial \tilde{P} \left( y \left( x, t \right), t \right)}{\partial t}
    =
    \cancel{-
    v * \frac{\partial \tilde{P} \left( y \left( x, t \right), t \right)}{\partial y}}
    +
    D * \frac{\partial^{2} \tilde{P} \left( y \left( x, t \right), t \right)}{\partial y^{2}}
\end{equation}
Ezt követően pedig megkapjuk a végleges egyenletünket:

\begin{equation}
    \frac{\partial \tilde{P} \left( y \left( x, t \right), t \right)}{\partial t}
    =
    D * \frac{\partial^{2} \tilde{P} \left( y \left( x, t \right), t \right)}{\partial y^{2}}
\end{equation}
Mely $\tilde{P}$-re vonatkozólag természetesen itt is megegyezik a driftmentes leírás Fokker--Planck-egyenletével. Ennek megoldása $\tilde{P} \left( y, t = 0 \right) = \delta \left( y \right)$ keződfeltétellel (mely ekvivalens a $P \left( x, t = 0 \right) = \delta \left( x \right)$ feltétellel) a már ismert Gauss-függvény:

\begin{equation}
    \tilde{P} \left( y, t \right) = \frac{1}{\sqrt{4 \pi D t}} * e^{-\tfrac{y^{2}}{4Dt}}
\end{equation}
Melyet az előző feladatban is szereplő változócserével visszaalakítva, megkapjuk az egyenletünk megoldását:

\begin{equation}
    \boxed{P \left( x, t \right) = \frac{1}{\sqrt{4 \pi D t}} * e^{-\tfrac{\left(x - vt\right)^{2}}{4Dt}}}
\end{equation}