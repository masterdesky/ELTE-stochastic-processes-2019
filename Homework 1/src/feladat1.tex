\section{}
\subsection*{A) feladatrész}
\quest{Próbáljunk emlékezni arra az eseményre, amivel kapcsolatban először gondoltunk véletlenszerűségre. Miért tekintettük az eseményt véletlennek, s mit gondoltunk a jelenség hátteréről?}
\\ \\

\subsection*{B) feladatrész}
\quest{Emlékezzünk olyan, az életünkben megtörtént eseményre, amikor kiszámoltunk valószínűségeket (az adott ismereteinkből kiindulva), s ezek a valószínűségek határozták meg a tetteinket!}
\\ \\

\subsection*{C) feladatrész}
\quest{Találjunk olyan véletlenszerű jelenséget környezetünkben, amelyre a Brown-mozgás típusú leírás jó közelítést adna!}
\\ \\
\begin{enumerate}
    \item A pénzügyi világ és a piac dinamikáját leíró matematikai modellek egyik megközelítési módja a Brown-mozgást/fehér zajt leíró matematikai formalizmus (pl. Langevin egyenlet\cite{2013ChPhL..30h8901T}) alkalmazása. A tőzsdei mozgásokat, értékpapírok, vagy valuták értékének változását és sok mást is modell szinten, alkalmas így leírni.
    \item 
\end{enumerate}