\section{}
\subsection*{A) feladatrész}
\quest{Próbáljunk emlékezni arra az eseményre, amivel kapcsolatban először gondoltunk véletlenszerűségre. Miért tekintettük az eseményt véletlennek, s mit gondoltunk a jelenség hátteréről?}
\\ \\
9-10 éves koromig bezárólag néha-néha édesapám szelvényeket vett az M1-en futó Luxor című szerencsejátékba. Ez egy szombat esti családi programként működött nálunk és ilyenkor mindegyikőnk kapott egy-egy szelvényt, amit ő maga kellett kitöltsön a játék során. Egy szelvényen két egymás alatti $5 \times 5$-ös négyzetben szerepeltek $1$-$90$-ig nyerőszámok, és egy szám többször is, de legfeljebb háromszor fordulhatott elő rajta. A számokat a játék során egyesével sorsolták, géppel. Ha egy játékos szelvényén szerepelt egy kihúzott szám, azt azon karikázással jelölte. A cél az volt, hogy az egyik négyzet külső keretében, vagy az azon belüli $3 \times 3$-as mezőben minden számot eltaláljunk. A játék minden héten az első főnyeremény értékű találatig folytatódott, mely egy egész $5 \times 5$-ös négyzet megtöltését jelentette. \\
Ezen keresztül találkoztam először a \q{véletlen} fogalmával - és egyben a kezdetekben így ismerkedtem meg jobban a számokkal is. Megértettem, hogy mit jelent a \q{véletlen húzás} és azt is persze, hogy hiába húznak ki akár $35$-$40$ számot is egy szerencsejáték során a $90$-ből, milyen kis valószínűséggel nyerhet bármit is az ember rajtuk.

\subsection*{B) feladatrész}
\quest{Emlékezzünk olyan, az életünkben megtörtént eseményre, amikor kiszámoltunk valószínűségeket (az adott ismereteinkből kiindulva), s ezek a valószínűségek határozták meg a tetteinket!}
\\ \\
Ha az előző példához hasonlóan a legelső ezzel kapcsolatos emlékemet idézem fel, az általános iskolához kötődik, ahol $3$-$4$. osztályos korunkban nagyon sokat játszottunk kő-papír-ollót szünetekben. Kisebb koromban rengeteget néztem a National Geographic Channel-t, és egy ott látott műsor után - ami a szerencsejátékról és valószínűségekről szólt - a fejembe vettem, hogy \q{taktikázni} fogok a jövőben a játékok alkalmával. Megfigyeltem - talán a műsor tanácsára -, hogy az osztálytársaim nagyon ritkán mutatják kétszer ugyanazt a jelet egymás után. Így rájöttem, hogy érdemes ez alapján gondolkodni: olyan jel mutatásával lesz a legnagyobb esélyem gyakorlatban a nyerésre, amit az ellenfél előző lépése legyőzne, de a fennmaradó kettő közül az egyiket legyőzi, a másikkal pedig döntetlent játszik. \\
Természetesen ez minden esetben fennáll a kő-papír-olló szabályai szerint. Elég csupán arra figyelnem, hogy az ellenfél előző lépése melyik jelet üti a lehetséges $3$ közül. Szigorú értelemben itt nem \q{kiszámoltam} a valószínűségeket, csupán figyelembe vettem, hogy mi a \q{valószínűbb} esemény.
\\ \\
Elméleti síkon természetesen nem korrekt a gondolkodás, mivel mind a kő, a papír és az olló mutatása egymástól független esemény. Egy jelet mindig azonos eséllyel követ egy tetszőleges másik, tehát minden azonosan hosszú sorozat előfordulási valószínűsége megegyezik.

\subsection*{C) feladatrész}
\quest{Találjunk olyan véletlenszerű jelenséget környezetünkben, amelyre a Brown-mozgás típusú leírás jó közelítést adna!}

\begin{enumerate}
    \item A pénzügyi világ és a piac dinamikáját leíró matematikai modellek egyik megközelítési módja a Brown-mozgást/fehér zajt leíró matematikai formalizmus (pl. Langevin egyenlet\cite{2013ChPhL..30h8901T}) alkalmazása. A tőzsdei mozgásokat, értékpapírok, vagy valuták értékének változását és sok mást is alkalmas így leírni, ugyanis az előrejelzési modelleket nagyban javíthatja.
    \item Állatpopulációk, vagy egyes egyedek mozgásának leírásához a sztochasztikus differenciálegyenletek használata (mint amilyen pl. a Brown-mozgás leírása is) megszokottnak számít az etológiában. Egy állat mozgása általában véletlenszerű, grafikusan is kísértetiesen hasonlít a részecskék Brown-mozgását szemléltető ábrákhoz.\cite{brillinger2003simulating}\cite{pozdnyakov2014modeling}\cite{bearup2016revisiting}
\end{enumerate}