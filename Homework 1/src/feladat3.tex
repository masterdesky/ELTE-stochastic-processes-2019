\section{} \label{sec:3}
\subsection{Szimmetrikus - driftmentes - rendszer} \label{subsec:3.1}
\quest{Egydimenziós mozgást végző részecske $\tau$ időközönként véletlen irányú erő hatására előző helyzetétől $l$ távolságra ugrik (egyenlő $p_{+} = p_{-} = \frac{1}{2}$ valószínűséggel jobbra vagy balra). A részecske az $x_{0} = 0$ pontból indul. \\
Határozzuk meg a $t = N_{\tau}$ idő alatti elmozdulás és az elmozdulás négyzetének átlagát, $\left< x_{r} \right>$-t és $\left< x_{r}^{2} \right>$-t}
\\ \\
Ismert, hogy az $x_{r}$ elmozdulás várható értéke

\begin{equation} \label{eq:6}
    \left< x_{r} \right>
    =
    \int_{\ -\infty}^{\ \infty} x * P \left( x, t \right)\ dx.
\end{equation}
Míg az elmozdulás négyzet várható értéke

\begin{equation} \label{eq:7}
    \left< x_{r}^{2} \right>
    =
    \int_{\ -\infty}^{\ \infty} x^{2} * P \left( x, t \right)\ dx.
\end{equation}
Jelölje az $P \left( x, t \right)$ az $x$ pontban való tartózkodás valószínűségét a keresett $t = N_\tau$ idő után. Ekkor felírhatjuk a következőket\cite{randwalk}:

\begin{equation} \label{eq:8}
    P \left( x, t + \tau \right)
    =
    p_{-} * P \left( x - l, t \right) + p_{+} * P \left( x + l, t \right)
\end{equation}
Kiinduló feltevéseink közé tartozik, hogy $p_{-} = p_{+} = \frac{1}{2}$, tehát a rendszer szimmetrikus. Ekkor a (\ref{eq:8})-as egyenlet a következőképp alakul:

\begin{equation} \label{eq:9}
    P \left( x, t + \tau \right)
    =
    \frac{1}{2} * P \left( x - l, t \right) + \frac{1}{2} * P \left( x + l, t \right)
\end{equation}
Ezt a differencia egyenletet a Kramers--Moyal-sorfejtés segítségével alakítjuk át a $P \left( x, t \right)$-re vonatkozó differenciálegyenletre, melyet Fokker--Planck-egyenletnek nevezünk\cite{2007cond.mat..1242G}. Első lépésben vonjunk ki mindkét oldalból $P \left( x, t \right)$, amiket aztán $\tau$ és $l$, $0$-hoz történő közelítésével sorbafejtünk $t$ és $x$ szerint:

\begin{equation} \label{eq:10}
    P \left( x, t + \tau \right) - P \left( x, t \right)
    =
    \frac{1}{2} * \left[ P \left( x - l, t \right) - P \left( x, t \right) \right] + \frac{1}{2} * \left[ P \left( x + l, t \right) - P \left( x, t \right) \right]
\end{equation}
A fentiek alapján alakítsuk át ezeket: a bal oldalt fejtsük sorba $t$, a jobb oldalt pedig $x$ szerint. A $t$ szerintinél az első, az $x$ szerintinél pedig a második rendig fejtsünk sorba:

\begin{equation} \label{eq:11}
    P \left( x, t + \tau \right) - P \left( x, t \right)
    =
    \tau \frac{\partial P \left( x, t \right)}{\partial t} + \mathcal{O} \left( \tau^{2} \right)
\end{equation}

\begin{equation} \label{eq:12}
    P \left( x - l, t \right) - P \left( x, t \right)
    =
    -l \frac{\partial P \left( x, t \right)}{\partial x} + \frac{1}{2} l^{2} \frac{\partial^{2} P \left( x, t \right)}{\partial x^{2}} + \mathcal{O} \left( l^{3} \right)
\end{equation}

\begin{equation} \label{eq:13}
    P \left( x + l, t \right) - P \left( x, t \right)
    =
    +l \frac{\partial P \left( x, t \right)}{\partial x} + \frac{1}{2} l^{2} \frac{\partial^{2} P \left( x, t \right)}{\partial x^{2}} + \mathcal{O} \left( l^{3} \right)
\end{equation}
A kapott eredményt helyettesítsük be az eredeti, (\ref{eq:10})-es egyenletbe:

\begin{align} \label{eq:14}
    &\tau \frac{\partial P \left( x, t \right)}{\partial t} + \mathcal{O} \left( \tau^{2} \right) = \nonumber \\
    &=
    \frac{1}{2} * \left[ -l \frac{\partial P \left( x, t \right)}{\partial x} + \frac{1}{2} l^{2} \frac{\partial^{2} P \left( x, t \right)}{\partial x^{2}} + \mathcal{O} \left( l^{3} \right) \right] + \nonumber \\
    &+
    \frac{1}{2} * \left[ l \frac{\partial P \left( x, t \right)}{\partial x} + \frac{1}{2} l^{2} \frac{\partial^{2} P \left( x, t \right)}{\partial x^{2}} + \mathcal{O} \left( l^{3} \right) \right]
\end{align}
A sorfejtés elhanyagolhatóan kicsi tagjait kihagyva:

\begin{equation} \label{eq:15}
    \tau \frac{\partial P \left( x, t \right)}{\partial t}
    =
    \frac{1}{2} * \left[ -l \frac{\partial P \left( x, t \right)}{\partial x} + \frac{1}{2} l^{2} \frac{\partial^{2} P \left( x, t \right)}{\partial x^{2}} \right] + \frac{1}{2} * \left[ l \frac{\partial P \left( x, t \right)}{\partial x} + \frac{1}{2} l^{2} \frac{\partial^{2} P \left( x, t \right)}{\partial x^{2}} \right]
\end{equation}
Rendezésnél az $x$-ben lineáris tagok kiesnek. A maradékot átosztva $\tau$-val, megkapjuk a Fokker--Planck-egyenletet:

\begin{equation} \label{eq:16}
    \tau \frac{\partial P \left( x, t \right)}{\partial t}
    =
    \frac{1}{2} * \left[ \cancel{-l \frac{\partial P \left( x, t \right)}{\partial x}} + \frac{1}{2} l^{2} \frac{\partial^{2} P \left( x, t \right)}{\partial x^{2}} + \cancel{l \frac{\partial P \left( x, t \right)}{\partial x}} + \frac{1}{2} l^{2} \frac{\partial^{2} P \left( x, t \right)}{\partial x^{2}} \right]
\end{equation}

\begin{equation} \label{eq:17}
    \frac{\partial P \left( x, t \right)}{\partial t} 
    =
    \frac{l^{2}}{2 \tau} * \frac{\partial^{2} P \left( x, t \right)}{\partial x^{2}}
    =
    D * \frac{\partial^{2} P \left( x, t \right)}{\partial x^{2}}
\end{equation}
Az ebben az egyenletben megjelenő $\frac{l^{2}}{2 \tau} = D$ tagot nevezzük a rendszer \emph{diffúziós együtthatójának}. Olyan esetben, amikor $p_{-} \neq p_{+}$, akkor a első rendű tagok is bent maradnak, megszorozva egy $- \left( p_{-} - p_{+} \right) \frac{l}{\tau} = - v$ együtthatóval, melyet a rendszer \emph{driftjének}, vagy \emph{sodródásának} hívunk.
\\ \\
A (\ref{eq:17})-es differenciálegyenletet a $P \left( x, t=0 \right) = \delta \left( x \right)$ kezdőfeltétellel oldjuk meg. Ennek megoldása ismert, ez a Gauss függvény:

\begin{equation} \label{eq:18}
    P \left( x, t \right) = \frac{1}{\sqrt{4 \pi D t}} * e^{-\tfrac{x^{2}}{4Dt}}
\end{equation}
Ennek a felhasználásával pedig megadhatjuk a keresett $\left< x_{r} \right>$ és $\left< x_{r}^{2} \right>$ várható értékeket a (\ref{eq:6})-os és (\ref{eq:7})-es egyenletek alapján:

\begin{equation} \label{eq:19}
    \left< x_{r} \right>
    =
    \int_{\ -\infty}^{\ \infty} x * \frac{1}{\sqrt{4 \pi D t}} * e^{-\tfrac{x^{2}}{4Dt}}\ dx
    =
    \frac{1}{\sqrt{4 \pi D t}} * \int_{\ -\infty}^{\ \infty} x * e^{-\tfrac{x^{2}}{4Dt}}\ dx
\end{equation}

\begin{equation} \label{eq:20}
    \left< x_{r}^{2} \right>
    =
    \int_{\ -\infty}^{\ \infty} x^{2} * \frac{1}{\sqrt{4 \pi D t}} * e^{-\tfrac{x^{2}}{4Dt}}\ dx
    =
    \frac{1}{\sqrt{4 \pi D t}} * \int_{\ -\infty}^{\ \infty} x^{2} * e^{-\tfrac{x^{2}}{4Dt}}\ dx
\end{equation}
Végezzük el a következő változócserét:

\begin{equation*}
    y := \frac{x}{\sqrt{4Dt}} \quad \to \quad x = y * \sqrt{4Dt}
\end{equation*}
\begin{equation*}
    \frac{dy}{dx} = \frac{1}{\sqrt{4Dt}} \quad \to \quad dx = \sqrt{4Dt}\, dy
\end{equation*}
Majd helyettesítsünk be a fenti (\ref{eq:19})-es és (\ref{eq:20})-as egyenletekbe:

\begin{align} \label{eq:21}
    \left< x_{r} \right>
    &=
    \frac{1}{\sqrt{4 \pi D t}} * \int_{\ -\infty}^{\ \infty} y * \sqrt{4Dt} * e^{-y^{2}} * \sqrt{4Dt}\, dy = \nonumber \\
    &=
    \frac{4Dt}{\sqrt{4 \pi D t}} * \int_{\ -\infty}^{\ \infty} y * e^{-y^{2}}\, dy
    =
    \frac{4Dt}{\sqrt{4 \pi D t}} * 0 = \doubleunderline{0}
\end{align}

\begin{align} \label{eq:22}
    \left< x_{r}^{2} \right>
    &=
    \frac{1}{\sqrt{4 \pi D t}} * \int_{\ -\infty}^{\ \infty} \left( y * \sqrt{4Dt} \right)^{2} * e^{-y^{2}} * \sqrt{4Dt}\, dy = \nonumber \\
    &=
    \frac{\left( 4Dt \right)^{\tfrac{3}{2}}}{\sqrt{4 \pi D t}} * \int_{\ -\infty}^{\ \infty} y^{2} * e^{-y^{2}}\, dy
    =
    \frac{4Dt}{\sqrt{\pi}} * \frac{\sqrt{\pi}}{2}
    =
    \doubleunderline{2Dt}
\end{align}

\subsection{Asszimetrikus - driftelő - rendszer} \label{subsec:3.2}
\quest{Vizsgáljuk a fenti problémát $p_{+} = 4p_{-}$ esetre és számítsuk ki az $\left< x_{r} \right>$, $\left< x_{r}^{2} \right>$ és a $\left< x_{r}^{2} \right> - \left< x_{r} \right>$ átlagokat!}
\\ \\
Nézzük azt a helyzetet, amikor a rendszerben van drift, tehát $p_{-} \neq p_{+}$, ahol most $p_{-} = \frac{1}{5}$ és $p_{+} = \frac{4}{5}$. Ebben az esetben a (\ref{eq:8})-as egyenletet a következőképp alakul:

\begin{equation} \label{eq:23}
    P \left( x, t + \tau \right)
    =
    \frac{1}{5} * P \left( x - l, t \right) + \frac{4}{5} * P \left( x + l, t \right)
\end{equation}
Elvégezve a (\ref{eq:10})-(\ref{eq:15}) egyenletekhez hasonlóan a Kramer--Moyal-sorfejtést, az elsőrendű tagok utána már nem esnek ki. Ekvivalensen a (\ref{eq:15})-ös egyenletben szereplő lépés itt most így fest:

\begin{equation} \label{eq:24}
    \tau \frac{\partial P \left( x, t \right)}{\partial t}
    =
    \frac{1}{5} * \left[ -l \frac{\partial P \left( x, t \right)}{\partial x} + \frac{1}{2} l^{2} \frac{\partial^{2} P \left( x, t \right)}{\partial x^{2}} \right] + \frac{4}{5} * \left[ l \frac{\partial P \left( x, t \right)}{\partial x} + \frac{1}{2} l^{2} \frac{\partial^{2} P \left( x, t \right)}{\partial x^{2}} \right]
\end{equation}
Ezt az egyenletet rendezve a következő alakot kapjuk az előző alfejezet végén leírtak alapján:

\begin{equation} \label{eq:25}
    \frac{\partial P \left( x, t \right)}{\partial t}
    =
    \frac{3}{5} * \frac{l}{\tau} \frac{\partial P \left( x, t \right)}{\partial x} + \frac{l^{2}}{2 \tau} * \frac{\partial^{2} P \left( x, t \right)}{\partial x^{2}}
    =
    - v \frac{\partial P \left( x, t \right)}{\partial x} + D * \frac{\partial^{2} P \left( x, t \right)}{\partial x^{2}}
\end{equation}
A kapott differenciálegyenletet az előzőekhez hasonlóan, szintén a $P \left( x, t=0 \right) = \delta \left( x \right)$ kezdőfeltétellel oldjuk meg. Ennek megoldásához bevezetjük a következő változócserét:

\begin{equation*}
    P \left( x, t \right)
    :=
    \tilde{P} \left( y \left( x, t \right), t \right)
    =
    \tilde{P} \left( x - vt, t \right)
\end{equation*}
Melyre a (\ref{eq:25})-ös egyenlet a következőképp módosul:

\begin{align} \label{eq:26}
    &\frac{\partial P \left( x, t \right)}{\partial t}
    =
    - v \frac{\partial P \left( x, t \right)}{\partial x} + D * \frac{\partial^{2} P \left( x, t \right)}{\partial x^{2}} \quad \to \nonumber \\
    \to \quad &
    \frac{\partial \tilde{P} \left( y \left( x, t \right), t \right)}{\partial t}
    =
    - v \frac{\partial \tilde{P} \left( y \left( x, t \right), t \right)}{\partial x} + D * \frac{\partial^{2} \tilde{P} \left( y \left( x, t \right), t \right)}{\partial x^{2}}
\end{align}
Bontsuk ki a parciális deriváltakat:

\begin{align}
    &\frac{\partial y \left( x, t \right)}{\partial t} * \frac{\partial \tilde{P} \left( y \left( x, t \right), t \right)}{\partial y}
    +
    \frac{\partial t}{\partial t} * \frac{\partial \tilde{P} \left( y \left( x, t \right), t \right)}{\partial t}
    = \nonumber \\
    =&
    -
    v * \frac{\partial y \left( x, t \right)}{\partial x} * \frac{\partial \tilde{P} \left( y \left( x, t \right), t \right)}{\partial y}
    +
    D * \partial_{x} \left( \frac{\partial y \left( x, t \right)}{\partial x} * \frac{\partial \tilde{P} \left( y \left( x, t \right), t \right)}{\partial y} \right)
\end{align}
\hrulefill
\begin{align}
    &\frac{\partial y \left( x, t \right)}{\partial t} * \frac{\partial \tilde{P} \left( y \left( x, t \right), t \right)}{\partial y}
    +
    \frac{\partial t}{\partial t} * \frac{\partial \tilde{P} \left( y \left( x, t \right), t \right)}{\partial t}
    = \nonumber \\
    =&
    -
    v * \frac{\partial y \left( x, t \right)}{\partial x} * \frac{\partial \tilde{P} \left( y \left( x, t \right), t \right)}{\partial y}
    +
    D * \partial_{x} \left( \frac{\partial \left( x - vt \right)}{\partial x} * \frac{\partial \tilde{P} \left( y \left( x, t \right), t \right)}{\partial y} \right)
\end{align}
\hrulefill
\begin{align}
    &\frac{\partial \left( x - vt \right)}{\partial t} * \frac{\partial \tilde{P} \left( y \left( x, t \right), t \right)}{\partial y}
    +
    1 * \frac{\partial \tilde{P} \left( y \left( x, t \right), t \right)}{\partial t}
    = \nonumber \\
    =&
    -
    v * \frac{\partial \left( x - vt \right)}{\partial x} * \frac{\partial \tilde{P} \left( y \left( x, t \right), t \right)}{\partial y}
    +
    D * \frac{\partial \left( x - vt \right)}{\partial x} * \frac{\partial \left( x - vt \right)}{\partial x} * \frac{\partial^{2} \tilde{P} \left( y \left( x, t \right), t \right)}{\partial y^{2}}
\end{align}
\hrulefill
\begin{align}
    &\underbrace{\frac{\partial \left( x - vt \right)}{\partial t}}_{\text{\normalfont $= -v$}} * \frac{\partial \tilde{P} \left( y \left( x, t \right), t \right)}{\partial y}
    +
    1 * \frac{\partial \tilde{P} \left( y \left( x, t \right), t \right)}{\partial t}
    = \nonumber \\
    =&
    -
    v * \underbrace{\frac{\partial \left( x - vt \right)}{\partial x}}_{\text{\normalfont $= 1$}} * \frac{\partial \tilde{P} \left( y \left( x, t \right), t \right)}{\partial y}
    +
    D * \underbrace{\frac{\partial \left( x - vt \right)}{\partial x}}_{\text{\normalfont $= 1$}} * \underbrace{\frac{\partial \left( x - vt \right)}{\partial x}}_{\text{\normalfont $= 1$}} * \frac{\partial^{2} \tilde{P} \left( y \left( x, t \right), t \right)}{\partial y^{2}}
\end{align}
\hrulefill
\begin{equation}
    \cancel{-v * \frac{\partial \tilde{P} \left( y \left( x, t \right), t \right)}{\partial y}}
    +
    \frac{\partial \tilde{P} \left( y \left( x, t \right), t \right)}{\partial t}
    =
    \cancel{-
    v * \frac{\partial \tilde{P} \left( y \left( x, t \right), t \right)}{\partial y}}
    +
    D * \frac{\partial^{2} \tilde{P} \left( y \left( x, t \right), t \right)}{\partial y^{2}}
\end{equation}
Ezt követően pedig megkapjuk a végleges egyenletünket:

\begin{equation}
    \frac{\partial \tilde{P} \left( y \left( x, t \right), t \right)}{\partial t}
    =
    D * \frac{\partial^{2} \tilde{P} \left( y \left( x, t \right), t \right)}{\partial y^{2}}
\end{equation}
Mely $\tilde{P}$-re vonatkozólag megegyezik a driftmentes leírás Fokker--Planck-egyenletével. Ennek megoldása $\tilde{P} \left( y, t = 0 \right) = \delta \left( y \right)$ keződfeltétellel (mely ekvivalens a $P \left( x, t = 0 \right) = \delta \left( x \right)$ feltétellel) a már ismert Gauss-függvény:

\begin{equation}
    \tilde{P} \left( y, t \right) = \frac{1}{\sqrt{4 \pi D t}} * e^{-\tfrac{y^{2}}{4Dt}}
\end{equation}
Melyet változócserével visszaalakítva megkapjuk az egyenletünk megoldását:

\begin{equation}
    \boxed{P \left( x, t \right) = \frac{1}{\sqrt{4 \pi D t}} * e^{-\tfrac{\left( x - vt \right)^{2}}{4Dt}}}
\end{equation}

A (\ref{eq:6})-os és (\ref{eq:7})-es egyenletek alapján megadhatjuk a keresett $\left< x_{r} \right>$ és $\left< x_{r}^{2} \right>$, valamint az $\left< x_{r}^{2} \right> - \left< x_{r} \right>$ értékeket:

\begin{equation}
    \left< x_{r} \right>
    =
    \int_{- \infty}^{\infty} x P \left( x, t \right)\, dx
    =
    \int_{- \infty}^{\infty} x \frac{1}{\sqrt{4 \pi D t}} * e^{-\tfrac{\left( x - vt \right)^{2}}{4Dt}}\, dx
    =
    \frac{1}{\sqrt{4 \pi D t}} * \int_{- \infty}^{\infty} x * e^{-\tfrac{\left( x - vt \right)^{2}}{4Dt}}\, dx
\end{equation}
\begin{equation}
    \left< x_{r}^{2} \right>
    =
    \int_{- \infty}^{\infty} x^{2} P \left( x, t \right)\, dx
    =
    \int_{- \infty}^{\infty} x^{2} \frac{1}{\sqrt{4 \pi D t}} * e^{-\tfrac{\left( x - vt \right)^{2}}{4Dt}}\, dx
    =
    \frac{1}{\sqrt{4 \pi D t}} * \int_{- \infty}^{\infty} x^{2} * e^{-\tfrac{\left( x - vt \right)^{2}}{4Dt}}\, dx
\end{equation}
Vezessük be a következő változócserét:

\begin{equation*}
    y := \frac{x - vt}{\sqrt{4Dt}} \quad \to \quad x = y * \sqrt{4Dt} + vt
\end{equation*}
\begin{equation*}
    \frac{dy}{dx} = \frac{1}{\sqrt{4Dt}} \quad \to \quad dx = \sqrt{4Dt}\, dy
\end{equation*}
Ezt behelyettesítve a fentiekbe:

\begin{equation}
    \left< x_{r} \right>
    =
    \frac{1}{\sqrt{4 \pi D t}} * \int_{- \infty}^{\infty} \left ( y * \sqrt{4Dt} + vt \right) * e^{y^{2}} \sqrt{4Dt}\, dy
    =
    \frac{\sqrt{4Dt}}{\sqrt{4 \pi D t}} * \int_{- \infty}^{\infty} \left ( y * \sqrt{4Dt} + vt \right) * e^{y^{2}}\, dy
\end{equation}
\\
\begin{equation}
    \left< x_{r}^{2} \right>
    =
    \frac{1}{\sqrt{4 \pi D t}} * \int_{- \infty}^{\infty} \left ( y * \sqrt{4Dt} + vt \right)^{2} * e^{y^{2}} \sqrt{4Dt}\, dy
    =
    \frac{\sqrt{4Dt}}{\sqrt{4 \pi D t}} * \int_{- \infty}^{\infty} \left ( y * \sqrt{4Dt} + vt \right)^{2} * e^{y^{2}}\, dy
\end{equation}