\section{} \label{sec:3}
\quest{Használjuk a \ref{sec:2}. feladat eredményét, valamint a Brown mozgás Langevin féle leirásának eredményeképp kapott kifejezést a kolloidrészecskék diffúziós együtthatójára, s becsüljük meg az Avogadro számot! A kolloidrészecskék sűrűségét tekinthetjük vizhez közelinek, a hőmérsékletet pedig szobahőmérsékletnek.}
\\ \\
A diffúziós együtthatő Langevin-féle definíciója már szerepelt az előző feladat végén, azonban ezt tovább tudjuk alakítani, hogy az Avogadro szám kifejezésére alkalmas legyen.

\begin{equation} \label{eq:18}
    D
    =
    \frac{k_{B} T}{6 \pi \eta a}
    =
    \frac{\tfrac{R}{N_{A}} T}{6 \pi \eta a}
\end{equation}
Ebből pedig kifejezhetjük az Avogadro számot:

\begin{equation} \label{eq:19}
    N_{A} = \frac{R T}{6 D \pi \eta a}
\end{equation}
A megadott $a = 0.52\ \mu m$ részecskesugár felhasználásával és feltételezve, hogy a diffúzió $298\ K$, szobahőmérsékleten történik, ahol a víz viszkozitása $\eta = 0.89 * 10^{-3}\ Pa*s$, ki tudjuk számítani az $N_{A}$ különböző értékeit az összes fenti $D$-re kapott eredményre:

\begin{center}
\captionof{table}{Az Avogadro szám értéke}\label{tab:7}
\begin{tabular}{||c|c|c|c||}
    \toprule
                                                                  & 1. trajektória  & 2. trajektória  & 3. trajektória  \\ \hline \hline
    $N_{A} \left( D \left( \left< r^{2} \right> \right) \right)$  & $6.60*10^{23}$  & $1.90*10^{23}$  & $1.03*10^{24}$ \\ \hline
    $N_{A} \left( D \left( \left< x^{2} \right> \right) \right)$  & $1.09*10^{25}$  & $1.53*10^{25}$  & $1.67*10^{25}$ \\ \hline
    Irodalmi érték                                                & \multicolumn{3}{c||}{$6.022*10^{23}$} \\
    \bottomrule
\end{tabular}
\end{center}
