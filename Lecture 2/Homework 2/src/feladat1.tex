\section{} \label{sec:1}
\quest{A Perrin kísérlet megértéséhez először oldjuk meg a két-dimenziós Brown mozgás következő változatát:\\
Egy $l$ rácsállandójú négyzetrácson egy részecske $\tau$ időközönként, egyenlő valószínűséggel ugrik a négy szomszédos rácspont egyikébe, s az egymást követő lépések függetlenek egymástól. A részecske az $(x_{0} = 0, y_{0} = 0)$ pontból indul. \\
Határozzuk meg a $t = N \tau$ idő alatti várható elmozdulást, $\sqrt{\left< r^{2} \right>} = \sqrt{\left< x_{t}^{2} \right> + \left< y_{t}^{2} \right>}$-t!}

\subsection{Matematikai megfontolások}
Keressük az $\left< x_{t}^{2} \right>$ és az $\left< y_{t}^{2} \right>$ értékeket, amikből aztán megkaphatjuk a keresett $\sqrt{\left< r^{2} \right>}$ mennyiséget. Ezeknek definíciója ismert:

\begin{equation} \label{eq:1}
    \left< x_{t}^{2} \right>
    =
    \int_{\ -\infty}^{\ \infty} x^{2} * P \left( x, t \right)\ dx
\end{equation}
\begin{equation} \label{eq:2}
    \left< y_{t}^{2} \right>
    =
    \int_{\ -\infty}^{\ \infty} y^{2} * P \left( y, t \right)\ dy
\end{equation}
Jelen esetben a 2D Brown--mozgás leírását kell megadnunk, amely esetében egy részecske, az adott pontban történő tartózkodásának valószínűségét egy $P_{12} \left( x, y, t \right)$ mennyiség jellemzi. Ekkor az $\left< x^{2} * y^{2} \right>$ megadhatjuk a következő módon:

\begin{equation} \label{eq:3}
    \left< x^{2} * y^{2} \right>
    =
    \iint{x^{2} * y^{2} * P_{12} \left( x, y, t \right)\, dx\, dy}
\end{equation}
Mivel $x$ és $y$ független események, ezért $P_{12} \left( x, y, t \right)$ felbontható egy $P_{1} \left( x, t \right)$ és egy $P_{2} \left( y, t \right)$ függvény szorzatára a következőek alapján:

\begin{equation}
    P_{12} \left( x, y, t \right) = P_{1} \left( x, t \right) * P_{2} \left( y, t \right) \quad \quad \text{ha $x$ és $y$ független.}
\end{equation}
Ekkor a fenti (\ref{eq:3})-as egyenlet így alakul:

\begin{equation} \label{eq:4}
    \iint{x^{2} * y^{2} * P_{12} \left( x, y, t \right)\, dx\, dy}
    =
    \iint{x^{2} * y^{2} * P_{1} \left( x, t \right) * P_{2} \left( y, t \right)\, dx\, dy}
\end{equation}
Ezt azonos változók szerint ketté bonthatunk és bevezethetünk egy többi változótól független időfüggőséget is, mely végeredménye a következő:

\begin{equation} \label{eq:5}
    \left< x^{2} * y^{2} \right>
    =
    \iint{x^{2} * y^{2} * P_{1} \left( x, t \right) * P_{2} \left( y, t \right)\, dx\, dy}
    =
    \underbrace{\int{x^{2} * P_{1} \left( x, t \right)\, dx}}_{\text{\normalfont $\left< x^{2} \right>$}} * \underbrace{\int{y^{2} * P_{2} \left( y, t \right)\, dy}}_{\text{\normalfont $\left< y^{2} \right>$}}
\end{equation}
Ez azt jelenti, hogy a probléma felbontható az $x$ és az $y$ irányú 1D Brown-mozgás problémájára, ahol az elmozdulási valószínűségek mindkét esetben azonosak:

\begin{equation*}
    p_{x_{+}} = \frac{1}{4} \quad p_{x_{-}} = \frac{1}{4}
\end{equation*}
\begin{equation*}
    p_{y_{+}} = \frac{1}{4} \quad p_{y_{-}} = \frac{1}{4}
\end{equation*}

\subsection{Az átlagok vizsgálata}
Megoldandó tehát a $p_{x_{+}} = \frac{1}{4},\ p_{x_{-}} = \frac{1}{4}$ és a $p_{y_{+}} = \frac{1}{4},\ p_{y_{-}} = \frac{1}{4}$ elmozdulási valószínűségekkel rendelkező 1D Brown-mozgás feladatai. Az első házifeladatban\cite{hazi1} már bebizonyítottuk, hogy ha 1D-ban a két irányba történő elmozdulás valószínűsége azonos, akkor a rendszer sodródása $0$, valamint

\begin{equation}
    \left< x_{t} \right> = \left< y_{t} \right> = 0 \quad \quad \text{ha $p_{+} = p_{-}$.}
\end{equation}
A Fokker--Planck-egyenlet(ek), driftmentes rendszerben, 1D diffúzió esetén a következőképp fest(enek), szintén az első házifeladat alapján\cite{hazi1}:

\begin{equation}
    \frac{\partial P_{1} \left( x, t \right)}{\partial t} 
    =
    \frac{l^{2}}{2 \tau} * \frac{\partial^{2} P_{1} \left( x, t \right)}{\partial x^{2}}
    =
    D * \frac{\partial^{2} P_{1} \left( x, t \right)}{\partial x^{2}}
\end{equation}
\begin{equation}
    \frac{\partial P_{2} \left( y, t \right)}{\partial t} 
    =
    \frac{l^{2}}{2 \tau} * \frac{\partial^{2} P_{2} \left( y, t \right)}{\partial y^{2}}
    =
    D * \frac{\partial^{2} P_{2} \left( y, t \right)}{\partial y^{2}}
\end{equation}
Megjegyzendő, hogy a két egyenletben megjelenő $D$, diffúziós együttható egy és ugyanaz a mennyiség. A kezdőfeltétel mindkét esetben $x_{0} = 0$ és $y_{0} = 0$, amit felírhatunk a következőképpen:

\begin{equation*}
    P_{1} \left( x, t = 0 \right) = \delta \left( x \right)
\end{equation*}
\begin{equation*}
    P_{2} \left( y, t = 0 \right) = \delta \left( y \right)
\end{equation*}
Ezeknek megoldása ismert módon a Gauss-függvény:

\begin{equation}
    P_{1} \left( x, t \right) = \frac{1}{\sqrt{4 \pi D t}} * e^{-\tfrac{x^{2}}{4Dt}}
\end{equation}
\begin{equation}
    P_{2} \left( x, t \right) = \frac{1}{\sqrt{4 \pi D t}} * e^{-\tfrac{y^{2}}{4Dt}}
\end{equation}
A feladat megoldása végül a már egyszer megoldott, ugyanezen feladat alapján\cite{hazi1}:

\begin{equation}
    \left< x_{t}^{2} \right> = \left< y_{t}^{2} \right> = 2Dt
\end{equation}
Így ezen feladat megoldása ezek segítségével már megadható:

\begin{equation}
    \sqrt{\left< r_{t}^{2} \right>}
    =
    \sqrt{\left< x_{t}^{2} \right> + \left< y_{t}^{2} \right>}
    =
    \sqrt{2Dt + 2Dt}
    =
    \boxed{2 \sqrt{Dt}}
\end{equation}
Ami ismert módon a helyes eredmény a 2D diffúzió esetében. Általánosan azt mondhatjuk, hogy az elmozdulás négyzetének várható értéke az $n$ dimenziós izotróp diffúzió esetén\cite{diffusion}:

\begin{equation}
    \left< r_{t}^{2} \right> = 2nDt
\end{equation}